\section*{Домашнее задание 9}

% pmi, fall 2024

Дедлайн: 2024-12-09, 23:59.

\newcommand{\dBeta}{\mathrm{Beta}}

\begin{enumerate}

\item Величина $X$ имеет стандартное нормальное распределение $\cN(0;1)$, а $Y \sim \cN(10; 16)$.

\begin{enumerate}
    \item Найдите функцию производящую моменты $X$.
    \item Найдите $\E(X^{2024})$.
    \item Найдите $\E(\cos(aX))$.
    \item Найдите вероятность $\P(Y > 20)$ с помощью таблиц или компьютера. 
    \item Найдите число $a$ такое, что $\P(Y \in [10 - a; 10 + a]) = 0.7$ с помощью таблиц или компьютера.
\end{enumerate}

\item Величины $X_1$ и $X_2$ независимы и равномерно распределены на отрезке $[0;1]$, $Y_1 = R \cos \alpha$, $Y_2 = R \sin \alpha$, где
$R = \sqrt{-2\ln U_1}$, $\alpha = 2\pi U_2$.

\begin{enumerate}
    \item Найдите совместную функцию плотности вектора $Y = (Y_1, Y_2)$.
    \item Как распределены величины $Y_1$ и $Y_2$? Независимы ли они?
\end{enumerate}
Запишем вектор $Y$ как вектор-столбец и рассмотрим вектор $W = A \cdot Y$, где $A = \begin{pmatrix}
    5 & -2 \\
    3 & 9 \\
\end{pmatrix}$.
\begin{enumerate}[resume]
    \item Найдите ковариационную матрицу случайного вектора $W$. 
    \item Найдите совместную функцию плотности вектора $W$ и запишите её с помощью матрицы $A$. 
\end{enumerate}

\item Перед Алексеем Ивановичем три игровых автомата «однорукий бандит». 
Каждый «бандит» при игре против него приносит либо один фридрихсдор, либо ничего. 
Вероятности выигрыша равны $p_1$, $p_2$, $p_3$ и неизвестны Алексею.

После окончания игры номер $t$, для выбора «бандита»-противника на $t+1$-ю игру,
Алексей использует следующее правило.

Он генерирует три независимых бета-распределенных случайных величины, $R_i \sim \dBeta(1 + W_{it}, 1 + L_{it})$.
Здесь $W_{it}$ и $L_{it}$ — текущее количество выигрышей и проигрышей на $i$-м «бандите».
Для следующей партии Алексей Иванович выбирает того «бандита», у которого величина $R_i$ оказалась выше. 

\begin{enumerate}
    \item С помощью $10^4$ симуляций оцените ожидаемый выигрыш Алексея за $200$ партий при $p_1 = 0.3$, $p_2 = 0.4$, $p_3 = 0.5$.
\end{enumerate}

Полина тоже любит играть с «однорукими бандитами». 
Отыграв партию номер $t$ она выбирает для следующей партии того бандита, у которого больше величина
\[ 
\E(R_i \mid W_{it}, L_{it}) = \frac{1 + W_{it}}{1 + W_{it} + 1 + L_{it}}.
\]
Если оптимальных «бандитов» — несколько, то Полина равновероятно выбирает любого из них.

\begin{enumerate}[resume]
    \item С помощью $10^4$ симуляций оцените ожидаемый выигрыш Полины за $200$ партий при $p_1 = 0.3$, $p_2 = 0.4$, $p_3 = 0.5$.
\end{enumerate}


\end{enumerate}
