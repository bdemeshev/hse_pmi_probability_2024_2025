\documentclass[12pt]{article}

% \usepackage{physics}

\usepackage{hyperref}
\hypersetup{
    colorlinks=true,
    linkcolor=blue,
    filecolor=magenta,      
    urlcolor=cyan,
    pdftitle={Overleaf Example},
    pdfpagemode=FullScreen,
    }

\usepackage{tikzducks}

\usepackage{tikz} % картинки в tikz
\usepackage{microtype} % свешивание пунктуации

\usepackage{array} % для столбцов фиксированной ширины

\usepackage{indentfirst} % отступ в первом параграфе

\usepackage{sectsty} % для центрирования названий частей
\allsectionsfont{\centering}

\usepackage{amsmath, amsfonts, amssymb} % куча стандартных математических плюшек

\usepackage{comment}

\usepackage[top=2cm, left=1.2cm, right=1.2cm, bottom=2cm]{geometry} % размер текста на странице

\usepackage{lastpage} % чтобы узнать номер последней страницы

\usepackage{enumitem} % дополнительные плюшки для списков
%  например \begin{enumerate}[resume] позволяет продолжить нумерацию в новом списке
\usepackage{caption}

\usepackage{url} % to use \url{link to web}


\newcommand{\smallduck}{\begin{tikzpicture}[scale=0.3]
    \duck[
        cape=black,
        hat=black,
        mask=black
    ]
    \end{tikzpicture}}

\usepackage{fancyhdr} % весёлые колонтитулы
\pagestyle{fancy}
\lhead{}
\chead{}
\rhead{Домашние задания для самураев}
\lfoot{}
\cfoot{}
\rfoot{}

\renewcommand{\headrulewidth}{0.4pt}
\renewcommand{\footrulewidth}{0.4pt}

\usepackage{tcolorbox} % рамочки!

\usepackage{todonotes} % для вставки в документ заметок о том, что осталось сделать
% \todo{Здесь надо коэффициенты исправить}
% \missingfigure{Здесь будет Последний день Помпеи}
% \listoftodos - печатает все поставленные \todo'шки


% более красивые таблицы
\usepackage{booktabs}
% заповеди из докупентации:
% 1. Не используйте вертикальные линни
% 2. Не используйте двойные линии
% 3. Единицы измерения - в шапку таблицы
% 4. Не сокращайте .1 вместо 0.1
% 5. Повторяющееся значение повторяйте, а не говорите "то же"


\setcounter{MaxMatrixCols}{20}
% by crazy default pmatrix supports only 10 cols :)


\usepackage{fontspec}
\usepackage{libertine}
\usepackage{polyglossia}

\setmainlanguage{russian}
\setotherlanguages{english}

% download "Linux Libertine" fonts:
% http://www.linuxlibertine.org/index.php?id=91&L=1
% \setmainfont{Linux Libertine O} % or Helvetica, Arial, Cambria
% why do we need \newfontfamily:
% http://tex.stackexchange.com/questions/91507/
% \newfontfamily{\cyrillicfonttt}{Linux Libertine O}

\AddEnumerateCounter{\asbuk}{\russian@alph}{щ} % для списков с русскими буквами
\setlist[enumerate, 2]{label=\asbuk*),ref=\asbuk*}

%% эконометрические сокращения
\DeclareMathOperator{\Cov}{\mathbb{C}ov}
\DeclareMathOperator{\Corr}{\mathbb{C}orr}
\DeclareMathOperator{\Var}{\mathbb{V}ar}
\DeclareMathOperator{\col}{col}
\DeclareMathOperator{\row}{row}
\DeclareMathOperator{\pCorr}{\mathrm{pCorr}}

\let\P\relax
\DeclareMathOperator{\P}{\mathbb{P}}

\let\H\relax
\DeclareMathOperator{\H}{\mathbb{H}}


\DeclareMathOperator{\E}{\mathbb{E}}
% \DeclareMathOperator{\tr}{trace}
\DeclareMathOperator{\card}{card}

\DeclareMathOperator{\Convex}{Convex}

\newcommand \cN{\mathcal{N}}
\newcommand \dN{\mathcal{N}}
\newcommand \dBin{\mathrm{Bin}}


\newcommand \RR{\mathbb{R}}
\newcommand \NN{\mathbb{N}}





\begin{document}

\section*{Домашнее задание 9}

% pmi, fall 2024

Дедлайн: 2024-12-09, 23:59.

\newcommand{\dBeta}{\mathrm{Beta}}

\begin{enumerate}

\item Величина $X$ имеет стандартное нормальное распределение $\cN(0;1)$, а $Y \sim \cN(10; 16)$.

\begin{enumerate}
    \item Найдите функцию производящую моменты $X$.
    \item Найдите $\E(X^{2024})$.
    \item Найдите $\E(\cos(aX))$.
    \item Найдите вероятность $\P(Y > 20)$ с помощью таблиц или компьютера. 
    \item Найдите число $a$ такое, что $\P(Y \in [10 - a; 10 + a]) = 0.7$ с помощью таблиц или компьютера.
\end{enumerate}

\item Величины $X_1$ и $X_2$ независимы и равномерно распределены на отрезке $[0;1]$, $Y_1 = R \cos \alpha$, $Y_2 = R \sin \alpha$, где
$R = \sqrt{-2\ln U_1}$, $\alpha = 2\pi U_2$.

\begin{enumerate}
    \item Найдите совместную функцию плотности вектора $Y = (Y_1, Y_2)$.
    \item Как распределены величины $Y_1$ и $Y_2$? Независимы ли они?
\end{enumerate}
Запишем вектор $Y$ как вектор-столбец и рассмотрим вектор $W = A \cdot Y$, где $A = \begin{pmatrix}
    5 & -2 \\
    3 & 9 \\
\end{pmatrix}$.
\begin{enumerate}[resume]
    \item Найдите ковариационную матрицу случайного вектора $W$. 
    \item Найдите совместную функцию плотности вектора $W$ и запишите её с помощью матрицы $A$. 
\end{enumerate}

\item Перед Алексеем Ивановичем три игровых автомата «однорукий бандит». 
Каждый «бандит» при игре против него приносит либо один фридрихсдор, либо ничего. 
Вероятности выигрыша равны $p_1$, $p_2$, $p_3$ и неизвестны Алексею.

После окончания игры номер $t$, для выбора «бандита»-противника на $t+1$-ю игру,
Алексей использует следующее правило.

Он генерирует три независимых бета-распределенных случайных величины, $R_i \sim \dBeta(1 + W_{it}, 1 + L_{it})$.
Здесь $W_{it}$ и $L_{it}$ — текущее количество выигрышей и проигрышей на $i$-м «бандите».
Для следующей партии Алексей Иванович выбирает того «бандита», у которого величина $R_i$ оказалась выше. 

\begin{enumerate}
    \item С помощью $10^4$ симуляций оцените ожидаемый выигрыш Алексея за $200$ партий при $p_1 = 0.3$, $p_2 = 0.4$, $p_3 = 0.5$.
\end{enumerate}

Полина тоже любит играть с «однорукими бандитами». 
Отыграв партию номер $t$ она выбирает для следующей партии того бандита, у которого больше величина
\[ 
\E(R_i \mid W_{it}, L_{it}) = \frac{1 + W_{it}}{1 + W_{it} + 1 + L_{it}}.
\]
Если оптимальных «бандитов» — несколько, то Полина равновероятно выбирает любого из них.

\begin{enumerate}[resume]
    \item С помощью $10^4$ симуляций оцените ожидаемый выигрыш Полины за $200$ партий при $p_1 = 0.3$, $p_2 = 0.4$, $p_3 = 0.5$.
\end{enumerate}


\end{enumerate}


\end{document}
