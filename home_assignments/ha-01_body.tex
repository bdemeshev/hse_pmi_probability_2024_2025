\section*{Домашнее задание 1}

Дедлайн: 2024-09-16, 21:00.

\begin{enumerate}
\item Вася решает три задачи по теории вероятностей. 
Вероятности решить эти задачи равны $0.1$, $0.2$ и $0.3$. 
Решения задач никак не связаны между собой, знание ни одной из задач не помогает решить ни одну другую.
Обозначим буквой $N$ общее количество решенных задач. 

\begin{enumerate}
    \item Найдите все значения $N$ и их вероятности. 
    \item Найдите $\P(N > 1)$, $\E(N)$ и $\E(N^2)$.
\end{enumerate}

\item За работу Вася получает случайное целое количество $\xi$ баллов, равновероятно распределённое от $1$ до $n$. 

Найдите $\E(\xi)$, $\E(\xi^2)$, $\E(\xi^3)$.

\item Берём набор данных по ссылке 

\url{https://github.com/bdemeshev/hse_pmi_probability_2024_2025/raw/main/home_assignments/ha01_data.csv}.

Здесь две переменных: $y_i$ — количество просмотренных Васей рилзов в день $i$ 
и бинарная переменная $x_i$ ($x_i = A$ — обычный день, $x_i = B$ — день дедлайна по теории вероятностей).

Рассмотрим две гипотезы. 
Нулевая гипотеза $H_0$: приближение дедлайна по вероятностям никак не влияет на количество просмотренных рилзов.
Альтернативная гипотеза $H_1$: приближение дедлайна в среднем снижает количество просмотренных рилзов. 

\begin{enumerate}
    \item Посчитайте фактическое значение статистики $S = \bar y_B - \bar y_A$.
    \item Предполагая, что $H_0$ верна, сгенерируйте $10000$ случайных перестановок меток $x$ и для 
    каждой перестановки посчитайте значение статистики $S^{\text{new}} = \bar y_B^{\text{new}} - \bar y_A^{\text{new}}$.
    \item Оцените $p$-значение, в данном случае $p$-значение — это вероятность $\P(S^{\text{new}} \leq S \mid S, H_0)$.
    \item Для принятия решения, отвергать или нет $H_0$, мы используем уровень значимости $\alpha = 0.05$.
    Отвергаем ли мы $H_0$?
\end{enumerate}

\end{enumerate}

