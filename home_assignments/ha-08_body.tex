\section*{Домашнее задание 8}

% pmi, fall 2024

Дедлайн: 2024-12-03, 23:59.

\begin{enumerate}

\item Пара величин $(X, Y)$ имеет функцию плотности $f(x, y) = 2x^3 + y$ на квадрате $[0;1] \times [0;1]$ и $0$ за его пределами. 
\begin{enumerate}
    \item Найдите условную функцию плотности $f(y \mid x)$.
    \item Найдите частные функции плотности $f_X(x)$ и $f_Y(y)$.
    \item Найдите функцию плотности $f_W(w)$ и функцию распределения $F_W(w)$ величины $W = X - Y$. 
    \item Найдите ожидание $\E(X + 5Y)$ и дисперсию $\Var(X + 5Y)$.
    \item Найдите совместную функцию плотности пары $(V = 2X + 3Y, W = X - Y)$. Аккуратно укажите область, где новая плотность положительна. 
    \item Найдите условное ожидание $\E(Y \mid X = x)$ и условную дисперсию $\Var(Y \mid X = x)$.
\end{enumerate}

\item Рассмотрим пуассоновский поток снежинок $(X_t)$ падающих на раскрытую ладошку с интенсивностью $\lambda = 0.5$ снежинок в секунду.
\begin{enumerate}
%    \item Я только что раскрыл ладошку. Какова вероятность того, что следующая снежинка упадёт раньше, чем через три секунды?
    \item Какова вероятность того, что за $5$ секунд на ладошку упадёт не менее двух снежинок?
    \item Я только что раскрыл ладошку. Какова вероятность того, что следующие две снежинки упадут раньше, чем через три секунды?
    \item Выпишите функцию плотности времени $T$ от раскрытия ладошки до выпадения третьей снежинки. 
    \item Найдите $\E(T)$ и $\Var(T)$.
    \item Выпишите функцию плотности отношения $R$ времени выпадения третьей снежинки к времени выпадения десятой снежинки. 
    \item Найдите $\E(R)$ и $\Var(R)$.
    \item Найдите вероятность $\P(X_{10} = 5 \mid X_{4} = 1)$.
    \item Найдите условные ожидание $\E(X_{10} \mid X_{4} = 1)$ и дисперсию $\Var(X_{10} \mid X_4 = 1)$.
\end{enumerate}

\item Страховые случаи наступают согласно пуассоновскому потоку с интенсивностью $100$ случаев в месяц. 
Выплаты по каждому страховому случаю распределены независимо от других случаев и времени наступления равномерно $0$ до $1$ ундециллиона рублей. 

Проведите $10^4$ симуляций этого процесса длиной в $1$ месяц. 

\begin{enumerate}
    \item Постройте гистограмму суммарных выплат за $10$ дней. 
    \item Оцените вероятность того, что за $10$ дней придётся выплатить более $12$ ундециллионов рублей. 
    \item Оцените размер резерва, необходимый страховой компании для того, чтобы за месяц вероятность исчерпания этого резерва была равна $0.05$.
    \item Как изменятся ответы на вопросы (б) и (в), если месяц начался с понедельника, а в субботу и воскресенье интенсивность страховых случаев падает до $10$ случаев в месяц?
\end{enumerate}


\end{enumerate}
