\documentclass[12pt]{article}

% \usepackage{physics}

\usepackage{hyperref}
\hypersetup{
    colorlinks=true,
    linkcolor=blue,
    filecolor=magenta,      
    urlcolor=cyan,
    pdftitle={Overleaf Example},
    pdfpagemode=FullScreen,
    }

\usepackage{tikzducks}

\usepackage{tikz} % картинки в tikz
\usepackage{microtype} % свешивание пунктуации

\usepackage{array} % для столбцов фиксированной ширины

\usepackage{indentfirst} % отступ в первом параграфе

\usepackage{sectsty} % для центрирования названий частей
\allsectionsfont{\centering}

\usepackage{amsmath, amsfonts, amssymb} % куча стандартных математических плюшек

\usepackage{comment}

\usepackage[top=2cm, left=1.2cm, right=1.2cm, bottom=2cm]{geometry} % размер текста на странице

\usepackage{lastpage} % чтобы узнать номер последней страницы

\usepackage{enumitem} % дополнительные плюшки для списков
%  например \begin{enumerate}[resume] позволяет продолжить нумерацию в новом списке
\usepackage{caption}

\usepackage{url} % to use \url{link to web}


\newcommand{\smallduck}{\begin{tikzpicture}[scale=0.3]
    \duck[
        cape=black,
        hat=black,
        mask=black
    ]
    \end{tikzpicture}}

\usepackage{fancyhdr} % весёлые колонтитулы
\pagestyle{fancy}
\lhead{}
\chead{}
\rhead{Домашние задания для самураев}
\lfoot{}
\cfoot{}
\rfoot{}

\renewcommand{\headrulewidth}{0.4pt}
\renewcommand{\footrulewidth}{0.4pt}

\usepackage{tcolorbox} % рамочки!

\usepackage{todonotes} % для вставки в документ заметок о том, что осталось сделать
% \todo{Здесь надо коэффициенты исправить}
% \missingfigure{Здесь будет Последний день Помпеи}
% \listoftodos - печатает все поставленные \todo'шки


% более красивые таблицы
\usepackage{booktabs}
% заповеди из докупентации:
% 1. Не используйте вертикальные линни
% 2. Не используйте двойные линии
% 3. Единицы измерения - в шапку таблицы
% 4. Не сокращайте .1 вместо 0.1
% 5. Повторяющееся значение повторяйте, а не говорите "то же"


\setcounter{MaxMatrixCols}{20}
% by crazy default pmatrix supports only 10 cols :)


\usepackage{fontspec}
\usepackage{libertine}
\usepackage{polyglossia}

\setmainlanguage{russian}
\setotherlanguages{english}

% download "Linux Libertine" fonts:
% http://www.linuxlibertine.org/index.php?id=91&L=1
% \setmainfont{Linux Libertine O} % or Helvetica, Arial, Cambria
% why do we need \newfontfamily:
% http://tex.stackexchange.com/questions/91507/
% \newfontfamily{\cyrillicfonttt}{Linux Libertine O}

\AddEnumerateCounter{\asbuk}{\russian@alph}{щ} % для списков с русскими буквами
\setlist[enumerate, 2]{label=\asbuk*),ref=\asbuk*}

%% эконометрические сокращения
\DeclareMathOperator{\Cov}{\mathbb{C}ov}
\DeclareMathOperator{\plim}{plim}
\DeclareMathOperator{\Corr}{\mathbb{C}orr}
\DeclareMathOperator{\Var}{\mathbb{V}ar}
\DeclareMathOperator{\col}{col}
\DeclareMathOperator{\row}{row}
\DeclareMathOperator{\pCorr}{\mathrm{pCorr}}

\let\P\relax
\DeclareMathOperator{\P}{\mathbb{P}}

\let\H\relax
\DeclareMathOperator{\H}{\mathbb{H}}


\DeclareMathOperator{\E}{\mathbb{E}}
% \DeclareMathOperator{\tr}{trace}
\DeclareMathOperator{\card}{card}

\DeclareMathOperator{\Convex}{Convex}

\newcommand \cN{\mathcal{N}}
\newcommand \cF{\mathcal{F}}
\newcommand \cH{\mathcal{H}}

\newcommand \dN{\mathcal{N}}
\newcommand \dBin{\mathrm{Bin}}
\newcommand \dExpo{\mathrm{Expo}}
\newcommand \dUnif{\mathrm{Unif}}


\newcommand \RR{\mathbb{R}}
\newcommand \NN{\mathbb{N}}





\begin{document}

\section*{Домашнее задание 13}

% pmi, fall 2024

Дедлайн: 2025-03-16, 23:59.

Оцениваемые задачи:

\begin{enumerate}

    
\item Неправильный кубик выпадает с вероятностью $0.5$ шестеркой вверх. 
Остальные пять граней выпадают равновероятно. 
Случайная величина $X$ — остаток от деления номера грани на два, $Y$ — остаток от деления номера грани на три. 
\begin{enumerate}
\item Найдите $\E(Y\mid X)$, $\Var(X\mid Y)$ и $\P(X = 1 \mid Y)$.
\item Найдите $\Cov(\E(Y\mid X),\E(X\mid Y))$, $\Cov(\E(Y\mid X),X)$.
\end{enumerate}


\item Цена литра молока, $X$, распределена равномерно на отрезке $[1;2]$. 
Количество молока, которое дает корова Мурка, $Y$, распределено экспоненциально с $\lambda=1$. 
Надои не зависят от цены. 
Величина $S$ — выручка кота Матроскина от продажи всего объема молока.


\begin{enumerate}
\item Найдите $\E(S \mid X)$, $\Var(S \mid X)$.
\item Найдите функцию плотности величины $\Var(S \mid X)$
\end{enumerate}


\end{enumerate}

Бесценные задачи в удовольствие:

\begin{enumerate}[resume]


\item Рассмотрим независимые равномерные случайные величины $X_{1}\sim \dUnif[0;1]$, $X_2 \sim \dUnif[-1;2]$ и $Y_i = X_i^2$.

Найдите $\E(X_1 \mid Y_1)$ и $\E(X_2 \mid Y_2)$.
    
    
\item  Величина $X$ равномерна на отрезке $[0;1]$. 
Определим событие $A=\{X>0.1\}$, величину $Y = X^2$ и сигма-алгебру $\cF = \sigma(A)$.

Найдите $\E(Y \mid \cF)$, $\E(I_A \mid \sigma(Y))$ и $\E(I_A + Y \mid Y-I_A)$.
    

\item Кот Матроскин ловит карасей до тех пор, пока не поймает карася длиной более полуметра. 
Длины карасей независимы и равномерны от 0 до 1 метра. 
Обозначим буквой $N$ количество пойманных карасей, а буквой $S$ — их суммарную длину. 

Найдите $\E(S \mid N)$, $\Var(S \mid N)$, $\E(S)$, $\Var(S)$.

\item Величины $X_1$, \ldots, $X_{100}$ независимы и равномерны на $[0;1]$. 
Обозначим $L=\max\{X_1, X_2, \ldots, X_80\}$, $R = \max\{X_{81}, X_{82}, \ldots, X_{100}\}$ и $M=\max\{X_1, \ldots, X_{100}\}$.


\begin{enumerate}
\item Найдите $\P(L>R \mid L)$, $\P(L>R \mid R)$ и $\P(L>R \mid M)$.
\item Найдите $\E(X_1 \mid L )$ и $\E(X_1 \mid \min \{ X_1, \ldots, X_{100}\})$.
\end{enumerate}

\end{enumerate}


\end{document}
