\documentclass[12pt]{article}

% \usepackage{physics}

\usepackage{hyperref}
\hypersetup{
    colorlinks=true,
    linkcolor=blue,
    filecolor=magenta,      
    urlcolor=cyan,
    pdftitle={Overleaf Example},
    pdfpagemode=FullScreen,
    }

\usepackage{tikzducks}

\usepackage{tikz} % картинки в tikz
\usepackage{microtype} % свешивание пунктуации

\usepackage{array} % для столбцов фиксированной ширины

\usepackage{indentfirst} % отступ в первом параграфе

\usepackage{sectsty} % для центрирования названий частей
\allsectionsfont{\centering}

\usepackage{amsmath, amsfonts, amssymb} % куча стандартных математических плюшек

\usepackage{comment}

\usepackage[top=2cm, left=1.2cm, right=1.2cm, bottom=2cm]{geometry} % размер текста на странице

\usepackage{lastpage} % чтобы узнать номер последней страницы

\usepackage{enumitem} % дополнительные плюшки для списков
%  например \begin{enumerate}[resume] позволяет продолжить нумерацию в новом списке
\usepackage{caption}

\usepackage{url} % to use \url{link to web}


\newcommand{\smallduck}{\begin{tikzpicture}[scale=0.3]
    \duck[
        cape=black,
        hat=black,
        mask=black
    ]
    \end{tikzpicture}}

\usepackage{fancyhdr} % весёлые колонтитулы
\pagestyle{fancy}
\lhead{}
\chead{}
\rhead{Домашние задания для самураев}
\lfoot{}
\cfoot{}
\rfoot{}

\renewcommand{\headrulewidth}{0.4pt}
\renewcommand{\footrulewidth}{0.4pt}

\usepackage{tcolorbox} % рамочки!

\usepackage{todonotes} % для вставки в документ заметок о том, что осталось сделать
% \todo{Здесь надо коэффициенты исправить}
% \missingfigure{Здесь будет Последний день Помпеи}
% \listoftodos - печатает все поставленные \todo'шки


% более красивые таблицы
\usepackage{booktabs}
% заповеди из докупентации:
% 1. Не используйте вертикальные линни
% 2. Не используйте двойные линии
% 3. Единицы измерения - в шапку таблицы
% 4. Не сокращайте .1 вместо 0.1
% 5. Повторяющееся значение повторяйте, а не говорите "то же"


\setcounter{MaxMatrixCols}{20}
% by crazy default pmatrix supports only 10 cols :)


\usepackage{fontspec}
\usepackage{libertine}
\usepackage{polyglossia}

\setmainlanguage{russian}
\setotherlanguages{english}

% download "Linux Libertine" fonts:
% http://www.linuxlibertine.org/index.php?id=91&L=1
% \setmainfont{Linux Libertine O} % or Helvetica, Arial, Cambria
% why do we need \newfontfamily:
% http://tex.stackexchange.com/questions/91507/
% \newfontfamily{\cyrillicfonttt}{Linux Libertine O}

\AddEnumerateCounter{\asbuk}{\russian@alph}{щ} % для списков с русскими буквами
\setlist[enumerate, 2]{label=\asbuk*),ref=\asbuk*}

%% эконометрические сокращения
\DeclareMathOperator{\Cov}{\mathbb{C}ov}
\DeclareMathOperator{\Corr}{\mathbb{C}orr}
\DeclareMathOperator{\Var}{\mathbb{V}ar}
\DeclareMathOperator{\col}{col}
\DeclareMathOperator{\row}{row}

\let\P\relax
\DeclareMathOperator{\P}{\mathbb{P}}

\DeclareMathOperator{\E}{\mathbb{E}}
% \DeclareMathOperator{\tr}{trace}
\DeclareMathOperator{\card}{card}

\DeclareMathOperator{\Convex}{Convex}

\newcommand \cN{\mathcal{N}}
\newcommand \dN{\mathcal{N}}
\newcommand \dBin{\mathrm{Bin}}


\newcommand \RR{\mathbb{R}}
\newcommand \NN{\mathbb{N}}





\begin{document}

\section*{Домашнее задание 5}


У этого задания нет дедлайна и за него нет оценки. 
Если очень хочется что-то куда-то загрузить, то можно отправить своему семинаристу мемасик по теории вероятностей :)


\begin{enumerate}
\item Случайная величина $X$ принимает значения $1$, $2$, $3$ и $4$ с вероятностями $0.1$, $0.2$, $0.3$, $0.4$.
\begin{enumerate}
    \item Нарисуйте функцию распределения величины $X$, $F_X(x)$.
    \item Какой вероятностный смысл имеет площадь над функцией распределения $F_X(x)$ на участке $x \in [0;\infty)$?
    \item Нарисуйте функцию распределения случайной величины $Y = F_X(X)$.
\end{enumerate}

\item Функция плотности случайной величины $Y$ равна $c y^2$ на отрезке $[0;2]$ и нулю иначе. 
\begin{enumerate}
    \item Найдите константу $c$. 
    \item Найдите функцию распределения $Y$.
    \item Найдите $\P(Y > 1)$, $\P(Y = 0.75)$, $\E(Y)$, $\E(Y^2)$.
    \item Найдите функцию производящую моменты $Y$, $m_Y(t)$.
    \item Найдите $\P(Y > 1.5 \mid Y > 1)$, $\E(Y \mid Y>1)$, $\E(Y^2 \mid Y > 1)$.
    \item Найдите функцию плотности величины $W = 1 / Y$.
\end{enumerate}

\item Случайная велина $U$ равномерна на отрезке $[0;10]$, $Y = \min \{U^2, 25\}$.
\begin{enumerate}
    \item Запишите вероятность $\P(Y \in [y; y + \Delta])$ с точностью до $o(\Delta)$.
    \item Найдите функцию распределения $Y$.
    \item Найдите $\P(Y > 10)$, $\E(Y)$, $\E(Y^2)$.
    \item Найдите $\P(Y > 10 \mid Y > 5)$, $\E(Y \mid Y>5)$, $\E(Y^2 \mid Y > 5)$.
\end{enumerate}
\end{enumerate}


\end{document}
