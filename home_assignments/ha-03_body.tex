\section*{Домашнее задание 3}

Дедлайн: 2024-09-30, 21:00.

\begin{enumerate}

\item В анкету включён вопрос, на который респонденты стесняются отвечать правдиво. 
Например, «Берёте ли Вы взятки?» или «Употребляете ли Вы наркотики?»
Чтобы стимулировать респондентов отвечать правдиво, используют следующий прием. 
Перед ответом на вопрос респондент в тайне от анкетирующего подкидывает один раз специальную монетку, на гранях которой написано «Да = А, Нет = Б»,
и «Да = Б, Нет = А». 
Ответ «Да» на нескромный вопрос является верным для доли $p$ всех людей. 
Монетка неправильная и выпадает стороной «Да= А, Нет = Б» с вероятностью $0.6$.

\begin{enumerate}
    \item Какова вероятность того, что ответ «Да» для данного индивида верен, если он написал «A» и следовал указаниям монетки?
    \item Какова вероятность того, что ответ «Да» для данного индивида верен, если он подбрасывал специальную монету 3 раза,
    следовал каждый раз предлагаемой кодировке и написал «А», «Б», «А»?
\end{enumerate}

\item Илон Маск изобрёл новый кубик под названием Model-6. 
Он взял правильный игральный кубик и правильный кубик с неподписанными гранями. 
Он подкинул шесть раз правильный игральный кубик и заполнил по очереди все грани изначально чистого кубика результатами бросков правильного кубика.
\begin{enumerate}
    \item Какова вероятность того, что в первом броске Model-6 выпало 6, если во втором броске Model-6 выпало 6?
    \item Зависимы ли результаты бросков Model-6?
    \item Чему равно ожидаемое количество шестёрок, выпавших в процессе изготовления Model-6, если при шести бросках Model-6 выпало три шестёрки?
\end{enumerate}

\item Камала Харрис подбрасывает кубик до первого выпадения восьмёрки.
Все грани кубика выпадают равновероятно, однако на его шести гранях написаны числа 3, 4, 5, 6, 7, 8.
Дональд Трамп подбрасывает правильный октаэдр до выпадения восьмёрки. 
На гранях октаэдра написаны числа от 1, 2, 3, 4, 5, 6, 7, 8.

\begin{enumerate}
    \item Постройте гистограмму числа бросков кубика по $B = 10000$ экспериментов. 
    \item Оцените безусловное математическое ожидание числа бросков кубика.
    \item Оцените безусловную вероятность окончания игры быстрее, чем за 5 бросков кубика.
    \item Постройте условную гистограмму числа бросков октаэдра, если известно что грани 1 и 2 не выпадали. 
    Общее количество экспериментов здесь должно быть таким, чтобы число число экспериментов, где не выпадали грани 1 и 2 оказалось равным $B = 10000$.
    \item Оцените условное ожидание числа бросков октаэдра, если грани 1 и 2 не выпадали. 
    \item Оцените условную вероятность окончания игры быстрее, чем за 5 бросков октаэдра, если грани 1 и 2 не выпадали.
\end{enumerate}

\end{enumerate}