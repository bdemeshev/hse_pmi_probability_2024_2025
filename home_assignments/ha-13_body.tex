\section*{Домашнее задание 13}

% pmi, fall 2024

Дедлайн: 2025-03-16, 23:59.

Оцениваемые задачи:

\begin{enumerate}

    
\item Неправильный кубик выпадает с вероятностью $0.5$ шестеркой вверх. 
Остальные пять граней выпадают равновероятно. 
Случайная величина $X$ — остаток от деления номера грани на два, $Y$ — остаток от деления номера грани на три. 
\begin{enumerate}
\item Найдите $\E(Y\mid X)$, $\Var(X\mid Y)$ и $\P(X = 1 \mid Y)$.
\item Найдите $\Cov(\E(Y\mid X),\E(X\mid Y))$, $\Cov(\E(Y\mid X),X)$.
\end{enumerate}


\item Цена литра молока, $X$, распределена равномерно на отрезке $[1;2]$. 
Количество молока, которое дает корова Мурка, $Y$, распределено экспоненциально с $\lambda=1$. 
Надои не зависят от цены. 
Величина $S$ — выручка кота Матроскина от продажи всего объема молока.


\begin{enumerate}
\item Найдите $\E(S \mid X)$, $\Var(S \mid X)$.
\item Найдите функцию плотности величины $\Var(S \mid X)$
\end{enumerate}


\end{enumerate}

Бесценные задачи в удовольствие:

\begin{enumerate}[resume]


\item Рассмотрим независимые равномерные случайные величины $X_{1}\sim \dUnif[0;1]$, $X_2 \sim \dUnif[-1;2]$ и $Y_i = X_i^2$.

Найдите $\E(X_1 \mid Y_1)$ и $\E(X_2 \mid Y_2)$.
    
    
\item  Величина $X$ равномерна на отрезке $[0;1]$. 
Определим событие $A=\{X>0.1\}$, величину $Y = X^2$ и сигма-алгебру $\cF = \sigma(A)$.

Найдите $\E(Y \mid \cF)$, $\E(I_A \mid \sigma(Y))$ и $\E(I_A + Y \mid Y-I_A)$.
    

\item Кот Матроскин ловит карасей до тех пор, пока не поймает карася длиной более полуметра. 
Длины карасей независимы и равномерны от 0 до 1 метра. 
Обозначим буквой $N$ количество пойманных карасей, а буквой $S$ — их суммарную длину. 

Найдите $\E(S \mid N)$, $\Var(S \mid N)$, $\E(S)$, $\Var(S)$.

\item Величины $X_1$, \ldots, $X_{100}$ независимы и равномерны на $[0;1]$. 
Обозначим $L=\max\{X_1, X_2, \ldots, X_80\}$, $R = \max\{X_{81}, X_{82}, \ldots, X_{100}\}$ и $M=\max\{X_1, \ldots, X_{100}\}$.


\begin{enumerate}
\item Найдите $\P(L>R \mid L)$, $\P(L>R \mid R)$ и $\P(L>R \mid M)$.
\item Найдите $\E(X_1 \mid L )$ и $\E(X_1 \mid \min \{ X_1, \ldots, X_{100}\})$.
\end{enumerate}

\end{enumerate}
