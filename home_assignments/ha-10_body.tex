\section*{Домашнее задание 10}

% pmi, fall 2024

Дедлайн: 2025-02-09, 23:59.

Оцениваемые задачи:

\begin{enumerate}

\item Величины $(X_i)$ независимы и одинаково распределены с ожиданием $\E(X_i) = \mu \neq 0$ и дисперсией $\Var(X_i) = \sigma^2$.
Используя арифметику пределов и закон больших чисел, найдите пределы:
\begin{enumerate}
    \item $\plim (X_1^2 + X_2^2 + \dots + X_n^2) / (X_1 + X_2 + \dots + X_n + \dots + X_{2n})$;
    \item $\plim \sum_{i=1}^n (X_i - \bar X_n)^2 / (n - 1)$;
    \item $\plim \bar X_n / ((\bar X_n)^2 + 1)$.
\end{enumerate}


\item Величины $(X_i)$ независимы и одинаково распределены с ожиданием $\E(X_i) = \mu$ и дисперсией $\Var(X_i) = \sigma^2$.
Используя центральную предельную теорему и леммы Слуцкого, найдите пределы по распределению последовательностей 
\begin{enumerate}
    \item $(\sum_{i=1}^n X_i  -  n\mu) /\sqrt{n}$;
    \item $(\bar X_n - \mu) / \sqrt{\sigma^2/n}$;
    \item $(\bar X_n - \mu) / \sqrt{\sum_{j=1}^n (X_j - \bar X_n)^2 / (n^2 - n)}$;
\end{enumerate}


\item Величины $(X_i)$, $(Y_i)$, $(Z_i)$ имеют стандартное нормальное распределение $\cN(0; 1)$ и независимы как внутри последовательностей, так и между последовательностями.
Построим последовательности $R_i = X_i / \sqrt{Y_i^2}$ и $L_i = X_i /\sqrt{(Y_i^2 + Z_i^2)/2}$.
Определим накопленные средние $\bar R_n = (R_1 + R_2 + \dots + R_n) / n$ и, аналогично, $\bar L_n$.

\begin{enumerate}
    \item Постройте на одном графике пять траекторий $\bar R_n$ как функции от $n$ для $n \in \{1, \dots, 100000\}$.
    \item Постройте на одном графике пять траекторий $\bar L_n$ как функции от $n$ для $n \in \{1, \dots, 100000\}$.
    \item Прокомментируйте словами разницу между траекториями $\bar L_n$ и $\bar R_n$.
    \item Вспомните закон больших чисел и предположите, чем может быть вызвана разница в характере траекторий.
    \item Если возможно, найдите $\E(R_i)$ и $\E(L_i)$.
\end{enumerate}

Примечание: здесь без доказательства можно пользоваться тем, что функция плотности $R_i$ равна $f(r) = 1/(\pi(1+r^2))$, 
а функция плотности $L_i$ равна $f(l) = 1 / (2 + l^2)^{3/2}$.


\end{enumerate}

Бесценные задачи just for fun:

\begin{enumerate}[resume]
    \item Величины $(X_i)$, $(Y_i)$, $(Z_i)$ независимы и  имеют стандартное нормальное распределение $\cN(0; 1)$,
    $R_i = X_i / \sqrt{Y_i^2}$ и $L_i = X_i /\sqrt{(Y_i^2 + Z_i^2)/2}$.

    Докажите, что функция плотности $R_i$ равна $f(r) = 1/(\pi(1+r^2))$, а функция плотности $L_i$ равна $f(l) = 1 / (2 + l^2)^{3/2}$.

    \item Рассмотрим последовательность независимых величин $X_n \sim \dBeta(2n + 1, 5n + 10)$.
    \begin{enumerate}
        \item К чему сходится эта последовательность по вероятности?
        \item К чему сходится эта последовательность по распределению?
    \end{enumerate}

    \item Величины $U_i$ независимы и равномерны на отрезке $[0;1]$. 
    К чему и в каких смыслах (почти наверное, по вероятности, по распределению, $L^1$, $L^2$) сходится последовательность 
    \[
        X_n = \frac{\cos U_1 + \cos U_2 + \dots + \cos U_n}{2n + 1}?
    \]

    \item У Стива Джобса в гараже завалялось три неслучайных последовательности: $a_n = 1/n$, $b_n = 3 / (3 + n)$ и $c_n = 1/1^2 + 1/2^2 + \dots + 1/n^2$.
    Стив равновероятно выбирает одну из этих последовательностей и получает случайную последовательность $X_n$.

    \begin{enumerate}
        \item В каких смыслах (почти наверное, по вероятности, по распределению, $L^1$, $L^2$) и к чему сходится последовательность $X_n$?
        \item Запишите $\lim X_n$ в виде явной функции от $X_3$.
    \end{enumerate}

\end{enumerate}