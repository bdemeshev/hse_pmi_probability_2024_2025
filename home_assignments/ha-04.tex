\documentclass[12pt]{article}

% \usepackage{physics}

\usepackage{hyperref}
\hypersetup{
    colorlinks=true,
    linkcolor=blue,
    filecolor=magenta,      
    urlcolor=cyan,
    pdftitle={Overleaf Example},
    pdfpagemode=FullScreen,
    }

\usepackage{tikzducks}

\usepackage{tikz} % картинки в tikz
\usepackage{microtype} % свешивание пунктуации

\usepackage{array} % для столбцов фиксированной ширины

\usepackage{indentfirst} % отступ в первом параграфе

\usepackage{sectsty} % для центрирования названий частей
\allsectionsfont{\centering}

\usepackage{amsmath, amsfonts, amssymb} % куча стандартных математических плюшек

\usepackage{comment}

\usepackage[top=2cm, left=1.2cm, right=1.2cm, bottom=2cm]{geometry} % размер текста на странице

\usepackage{lastpage} % чтобы узнать номер последней страницы

\usepackage{enumitem} % дополнительные плюшки для списков
%  например \begin{enumerate}[resume] позволяет продолжить нумерацию в новом списке
\usepackage{caption}

\usepackage{url} % to use \url{link to web}


\newcommand{\smallduck}{\begin{tikzpicture}[scale=0.3]
    \duck[
        cape=black,
        hat=black,
        mask=black
    ]
    \end{tikzpicture}}

\usepackage{fancyhdr} % весёлые колонтитулы
\pagestyle{fancy}
\lhead{}
\chead{}
\rhead{Домашние задания для самураев}
\lfoot{}
\cfoot{}
\rfoot{}

\renewcommand{\headrulewidth}{0.4pt}
\renewcommand{\footrulewidth}{0.4pt}

\usepackage{tcolorbox} % рамочки!

\usepackage{todonotes} % для вставки в документ заметок о том, что осталось сделать
% \todo{Здесь надо коэффициенты исправить}
% \missingfigure{Здесь будет Последний день Помпеи}
% \listoftodos - печатает все поставленные \todo'шки


% более красивые таблицы
\usepackage{booktabs}
% заповеди из докупентации:
% 1. Не используйте вертикальные линни
% 2. Не используйте двойные линии
% 3. Единицы измерения - в шапку таблицы
% 4. Не сокращайте .1 вместо 0.1
% 5. Повторяющееся значение повторяйте, а не говорите "то же"


\setcounter{MaxMatrixCols}{20}
% by crazy default pmatrix supports only 10 cols :)


\usepackage{fontspec}
\usepackage{libertine}
\usepackage{polyglossia}

\setmainlanguage{russian}
\setotherlanguages{english}

% download "Linux Libertine" fonts:
% http://www.linuxlibertine.org/index.php?id=91&L=1
% \setmainfont{Linux Libertine O} % or Helvetica, Arial, Cambria
% why do we need \newfontfamily:
% http://tex.stackexchange.com/questions/91507/
% \newfontfamily{\cyrillicfonttt}{Linux Libertine O}

\AddEnumerateCounter{\asbuk}{\russian@alph}{щ} % для списков с русскими буквами
\setlist[enumerate, 2]{label=\asbuk*),ref=\asbuk*}

%% эконометрические сокращения
\DeclareMathOperator{\Cov}{\mathbb{C}ov}
\DeclareMathOperator{\Corr}{\mathbb{C}orr}
\DeclareMathOperator{\Var}{\mathbb{V}ar}
\DeclareMathOperator{\col}{col}
\DeclareMathOperator{\row}{row}

\let\P\relax
\DeclareMathOperator{\P}{\mathbb{P}}

\DeclareMathOperator{\E}{\mathbb{E}}
% \DeclareMathOperator{\tr}{trace}
\DeclareMathOperator{\card}{card}

\DeclareMathOperator{\Convex}{Convex}

\newcommand \cN{\mathcal{N}}
\newcommand \dN{\mathcal{N}}
\newcommand \dBin{\mathrm{Bin}}


\newcommand \RR{\mathbb{R}}
\newcommand \NN{\mathbb{N}}





\begin{document}

\section*{Домашнее задание 4}

Дедлайн: 2024-10-07, 21:00.

\begin{enumerate}
\item Случайные величины $(X_i)$ независимы и равновероятно принимают значения $0$ и $1$,
$S = X_1 + X_2 + \dots + X_n$, $W = (S - \E(S)) / \sqrt{n/4}$.

\begin{enumerate}
    \item Найдите производящую функции моментов $m_X(t)$ величины $X_i$.
    \item Найдите производящую функцию моментов $m_S(t)$ величины $S$.
    \item Найдите производящую функцию моментов $m_W(t)$ величины $W$.
    \item Найдите $\lim_{n\to\infty} m_W(t)$.
\end{enumerate}

\item Рассмотрим последовательность независимых биномиальных величин $X_k \sim \dBin(k, \lambda/k)$,
где $\lambda$ — параметр. 
\begin{enumerate}
    \item Найдите $\E(X_k)$, $\E(X_k^2)$ и предел $\lim_{k\to\infty} \E(X_k^2)$.
    \item Найдите предел вероятностей $\lim_{k\to\infty} \P(X_k = j)$. Верно ли, что $\sum_{j \geq 0} \lim_{k\to\infty} \P(X_k = j) = 1$?
    \item Найдите предел производящей функции моментов $\lim_{k\to\infty} m_k(t)$, где $m_k(t)= \E(\exp(tX_k))$.
\end{enumerate}

\item Алиса и Боб снова подкидывают монетку неограниченное число раз. 
Монетка выпадает решкой $H$ и орлом $T$ равновероятно. 
Алиса выигрывает, если последовательность $HHT$ выпадет раньше, а Боб — если раньше выпадет $HTH$.

Рассмотрим множество исходов этого эксперимента $\Omega = \{HHT, HTH, HHHT, THTH, THHT, \dots \}$
и производящую функцию исходов $f(H, T) = HHT + HTH + HHHT + THTH + THHT + \dots$.
Здесь аргументы $H$ и $T$ некоммутативны. 
Обозначим $X$ — количество решек $H$, $Y$ — количество орлов $T$.

\begin{enumerate}
    \item Укажите, как с помощью производных и подстановок раздобыть из функции $f(H, T)$ величины $\P(X = 10)$,
    $\P(X = 5, Y=5)$, $\E(X)$, $\E(X^3)$, $\E(X^2Y^3)$.
    \item С помощью метода первого шага составьте систему линейных уравнений, из которой можно найти $f(H, T)$. 
    \item Решите эту систему, предполагая коммутативность $H$ и $T$. 
    \item Завершите вычисление $\P(X = 10)$, $\P(X = 5, Y=5)$, $\E(X)$, $\E(X^3)$, $\E(X^2Y^3)$.
\end{enumerate}

Явное уточнение: конечно, в этой задаче можно использовать \verb|sympy| или другой пакет для символьного решения системы или вычисления производных. 

\end{enumerate}


\end{document}
