\section*{Домашнее задание 12}

% pmi, fall 2024

Дедлайн: 2025-03-12, 23:59.

Оцениваемые задачи:

\begin{enumerate}

    
\item Правильный кубик подбрасывается один раз.
Рассмотрим $Y$ — индикатор того, выпала ли четная грань
и $Z$ — индикатор того, выпало ли число больше двух.
\begin{enumerate}
\item Выпишите явно сигма-алгебру $\sigma(Y\cdot Z)$.
\item Выпишите явно сигма-алгебру $\sigma(Y, Z)$.
\end{enumerate}


\item Рассмотрим минимальную сигма-алгебру $\cF$ на $\RR$, в которую входят все конечные подмножества числовой прямой. 
\begin{enumerate}
    \item Приведите два примера бесконечных подмножеств, входящих в $\cF$.
    \item Приведите два примера числовых подмножеств, не входящих в $\cF$.
\end{enumerate}


\end{enumerate}

Бесценные задачи в удовольствие:

\begin{enumerate}[resume]


\item Сколько существует сигма-алгебр на множестве $\Omega$ из четырёх элементов?

\item Будем обозначать количество элементов множества с помощью $\card A$.
Рассмотрим подмножества натуральных чисел, $A \subseteq \mathbb{N}$.
Определим для подмножества плотность Чезаро (Cesaro density),
\[
\gamma(A)=\lim_{n\to \infty}\frac{\card (A\cap \{1,2,3, \ldots,n\})}{n}
\]
в тех случаях, когда этот предел существует.

Плотность Чезаро показывает, какую «долю» от всех натуральных чисел составляет указанное подмножество.
Обозначим с помощью $\cH$ все подмножества, имеющие плотность Чезаро.

Является ли набор $\cH$ сигма-алгеброй?

\item Случайные величины $(X_n)$ независимы и равновероятно равны $\pm 1$.
Накопленную сумму обозначим $S_n = X_1 + X_2 + \dots + X_n$.

\begin{enumerate}
    \item Сколько событий сигма-алгебрах $\sigma(S_5)$? $\sigma(X_5)$?
    \item Как связаны между собой сигма-алгебры $\sigma(X_1, X_2, \dots, X_{100})$ и $\sigma(S_1, S_2, \dots, S_{100})$?
\end{enumerate}

\item В лесу есть три вида грибов: рыжики, лисички и мухоморы. 
Попадаются они равновероятно и независимо друг от друга. 
Маша нашла 100 грибов. 
Пусть $R$ — количество рыжиков, $L$ — количество лисичек, а $M$ — количество мухоморов среди найденных грибов.
\begin{enumerate}
\item Сколько элементов $\sigma(R - M)$?
\item Сколько элементов $\sigma(R, M)$?
\item Измерима ли $L$ относительно $\sigma(R, M)$?
\item Измерима ли $L$ относительно $\sigma(R - M)$?
\end{enumerate}


\end{enumerate}
