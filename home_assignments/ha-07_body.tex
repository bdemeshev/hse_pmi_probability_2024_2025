\section*{Домашнее задание 7}

Дедлайн: 2024-11-01.

Здесь $\H(Y \mid X)$ — это условная энтропия, а $\H(X, Y)$ — совместная энтропия. 
Будьте осторожны, некоторые авторы используют обозначение $\H(X, Y)$ для кросс-энтропии. 


\begin{enumerate}
\item Распределение вектора $(X, Y)$ задано таблицей

\begin{center}
    \begin{tabular}{lccc}
    	\toprule
    	    & $Y = 1$  & $Y = 2$  & $Y = 3$ \\
        \midrule
    	$X = 0$ & $0.2$  & $0.2$  & $0.1$ \\
        $X = 1$ & $0.5$  &  $0$   & $0$ \\
      \bottomrule
    \end{tabular}
\end{center}

\begin{enumerate}
    \item Найдите энтропии $\H(X)$, $\H(Y)$, $\H(X, Y)$.
    \item Найдите $\H(Y \mid X)$.
    \item Какое максимальное значение может принимать условная энтропия $\H(Y \mid X)$, 
    если $X$ принимает два значения, а $Y$ — три?
\end{enumerate}

\item Рассмотрим равномерное распределение на отрезке $[0; 1]$.

\begin{enumerate}
    \item Найдите энтропию равномерного распределения на отрезке $[0; 1]$.
    \item Докажите, что равномерное распределение имеет максимальную энтропию среди всех распределений на отрезке $[0; 1]$,
    имеющих функцию плотности. 
\end{enumerate}

Рассмотим распределение с функцией плотности $f(x) = \exp(-x^2/2)/\sqrt{2\pi}$ на числовой прямой. 
Кстати, оно называется \emph{стандартным нормальным}. 

\begin{enumerate}[resume]
    \item Найдите математическое ожидание и дисперсию данного распределения. 
    \item Найдите энтропию стандартного нормального распределения.     
    \item Докажите, что стандартное нормальное распределение имеет максимальную энтропию среди всех распределений с функцией плотности с нулевым математическим ожиданием и единичной дисперсией.
\end{enumerate}

Подсказка: можно без доказательства пользоваться тем, что $\int_{-\infty}^{+\infty} f(x) dx = 1$.

\item Для дискретных величин $X$ и $Y$ докажите или опровергните утверждения:
\begin{enumerate}
    \item $\H(X) + \H(Y \mid X) = \H(X, Y)$;
    \item $\H(X, Y) \geq \H(X)$;
    \item $\H(X^2) = \H(X)$;
\end{enumerate}


\end{enumerate}


