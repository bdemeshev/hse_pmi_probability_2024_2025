\section*{Домашнее задание 2}

Дедлайн: 2024-09-23, 21:00.

\begin{enumerate}
\item Монетка выпадает орлом $T$ с вероятностью $0.2$ и решкой $H$ — с вероятностью $0.8$.
Илон Маск подбрасывает её $100$ раз. 
За каждую выпавшую комбинацию $THT$ он получает $1\$$, а за каждую комбинацию $HHHHHH$ — платит $1\$$.

Чему равен ожидаемый выигрыш Маска в эту игру?

Уточнение: комбинации могут пересекаться, например, за $THTHT$ Маск получит $2\$$.

\item Бармен Огненной Зебры разбавляет каждую кружку пива независимо от других с общеизвестной вероятностью $p \in (0;1)$.
Ковбой Джо заходит в бар и первым делом сразу заказывает три кружки пива и выпивает их.
Затем Джо заказывает по две кружки пива за один раз. 

После 3-й, 5-й, 7-й, 9-й, 11-й и далее через каждые две кружки Джо прислушивается к своим ощущениям.
Если не менее двух кружек пива из последних трёх кружек разбавлены, то Джо разносит бар к чертям собачьим. 

\begin{enumerate}
    \item Сколько кружек пива в среднем успеет выпить Джо прежде чем разнесёт Огненную Зебру?
    \item Если все три последние кружки пива разбавлены, то Джо разносит не только Огненную Зебру, 
    но и всю прилежащую улицу. Какова вероятность данного сценария?
\end{enumerate}


\item Ретроградный Меркурий. 

Маша решает задачи по теории вероятностей как во время ретроградного Меркурия, так и без оного. 
Всего она решила $50$ задач, из них $11$ она решила правильно. 
Из $30$ решённых во время ретроградного Меркурия задач $S = 5$ были решены правильно. 

Рассмотрим две гипотезы. 
Нулевая гипотеза $H_0$: ретроградный Меркурий не оказывает влияния на вероятность решения задачи.
Альтернативная гипотеза $H_1$: ретроградный Меркурий снижает вероятность верно решить задачу. 


\begin{enumerate}
    \item Предполагая, что $H_0$ верна, сгенерируйте $10000$ случайных выборок. 
    В каждой выборке должно быть всего ровно $50$ задач, ровно $30$ задач должны приходится на ретроградный Меркурий,
    всего ровно $11$ задач должны быть решены верно. 
    Для каждой выборки посчитайте $S^{\text{new}}$ — количество верно решённых во время ретроградного Меркурия задач.
    \item Оцените $p$-значение по $10000$ экспериментов. 
    В данном случае $p$-значение — это вероятность $\P(S^{\text{new}} \leq S \mid S, H_0)$.
    \item Найдите точное $p$-значение в данной задаче. 
    \item Для принятия решения, отвергать или нет $H_0$, мы используем уровень значимости $\alpha = 0.05$.
    Отвергаем ли мы $H_0$?
\end{enumerate}

\end{enumerate}
