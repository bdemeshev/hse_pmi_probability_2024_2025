\section*{Домашнее задание 4}

Дедлайн: 2024-10-07, 21:00.

\begin{enumerate}
\item Случайные величины $(X_i)$ независимы и равновероятно принимают значения $0$ и $1$,
$S = X_1 + X_2 + \dots + X_n$, $W = (S - \E(S)) / \sqrt{n/4}$.

\begin{enumerate}
    \item Найдите производящую функции моментов $m_X(t)$ величины $X_i$.
    \item Найдите производящую функцию моментов $m_S(t)$ величины $S$.
    \item Найдите производящую функцию моментов $m_W(t)$ величины $W$.
    \item Найдите $\lim_{n\to\infty} m_W(t)$.
\end{enumerate}

\item Рассмотрим последовательность независимых биномиальных величин $X_k \sim \dBin(k, \lambda/k)$,
где $\lambda$ — параметр. 
\begin{enumerate}
    \item Найдите $\E(X_k)$, $\E(X_k^2)$ и предел $\lim_{k\to\infty} \E(X_k^2)$.
    \item Найдите предел вероятностей $\lim_{k\to\infty} \P(X_k = j)$. Верно ли, что $\sum_{j \geq 0} \lim_{k\to\infty} \P(X_k = j) = 1$?
    \item Найдите предел производящей функции моментов $\lim_{k\to\infty} m_k(t)$, где $m_k(t)= \E(\exp(tX_k))$.
\end{enumerate}

\item Алиса и Боб снова подкидывают монетку неограниченное число раз. 
Монетка выпадает решкой $H$ и орлом $T$ равновероятно. 
Алиса выигрывает, если последовательность $HHT$ выпадет раньше, а Боб — если раньше выпадет $HTH$.

Рассмотрим множество исходов этого эксперимента $\Omega = \{HHT, HTH, HHHT, THTH, THHT, \dots \}$
и производящую функцию исходов $f(H, T) = HHT + HTH + HHHT + THTH + THHT + \dots$.
Здесь аргументы $H$ и $T$ некоммутативны. 
Обозначим $X$ — количество решек $H$, $Y$ — количество орлов $T$.

\begin{enumerate}
    \item Укажите, как с помощью производных и подстановок раздобыть из функции $f(H, T)$ величины $\P(X = 10)$,
    $\P(X = 5, Y=5)$, $\E(X)$, $\E(X^3)$, $\E(X^2Y^3)$.
    \item С помощью метода первого шага составьте систему линейных уравнений, из которой можно найти $f(H, T)$. 
    \item Решите эту систему, предполагая коммутативность $H$ и $T$. 
    \item Завершите вычисление $\P(X = 10)$, $\P(X = 5, Y=5)$, $\E(X)$, $\E(X^3)$, $\E(X^2Y^3)$.
\end{enumerate}

Явное уточнение: конечно, в этой задаче можно использовать \verb|sympy| или другой пакет для символьного решения системы или вычисления производных. 

\end{enumerate}
