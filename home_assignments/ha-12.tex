\documentclass[12pt]{article}

% \usepackage{physics}

\usepackage{hyperref}
\hypersetup{
    colorlinks=true,
    linkcolor=blue,
    filecolor=magenta,      
    urlcolor=cyan,
    pdftitle={Overleaf Example},
    pdfpagemode=FullScreen,
    }

\usepackage{tikzducks}

\usepackage{tikz} % картинки в tikz
\usepackage{microtype} % свешивание пунктуации

\usepackage{array} % для столбцов фиксированной ширины

\usepackage{indentfirst} % отступ в первом параграфе

\usepackage{sectsty} % для центрирования названий частей
\allsectionsfont{\centering}

\usepackage{amsmath, amsfonts, amssymb} % куча стандартных математических плюшек

\usepackage{comment}

\usepackage[top=2cm, left=1.2cm, right=1.2cm, bottom=2cm]{geometry} % размер текста на странице

\usepackage{lastpage} % чтобы узнать номер последней страницы

\usepackage{enumitem} % дополнительные плюшки для списков
%  например \begin{enumerate}[resume] позволяет продолжить нумерацию в новом списке
\usepackage{caption}

\usepackage{url} % to use \url{link to web}


\newcommand{\smallduck}{\begin{tikzpicture}[scale=0.3]
    \duck[
        cape=black,
        hat=black,
        mask=black
    ]
    \end{tikzpicture}}

\usepackage{fancyhdr} % весёлые колонтитулы
\pagestyle{fancy}
\lhead{}
\chead{}
\rhead{Домашние задания для самураев}
\lfoot{}
\cfoot{}
\rfoot{}

\renewcommand{\headrulewidth}{0.4pt}
\renewcommand{\footrulewidth}{0.4pt}

\usepackage{tcolorbox} % рамочки!

\usepackage{todonotes} % для вставки в документ заметок о том, что осталось сделать
% \todo{Здесь надо коэффициенты исправить}
% \missingfigure{Здесь будет Последний день Помпеи}
% \listoftodos - печатает все поставленные \todo'шки


% более красивые таблицы
\usepackage{booktabs}
% заповеди из докупентации:
% 1. Не используйте вертикальные линни
% 2. Не используйте двойные линии
% 3. Единицы измерения - в шапку таблицы
% 4. Не сокращайте .1 вместо 0.1
% 5. Повторяющееся значение повторяйте, а не говорите "то же"


\setcounter{MaxMatrixCols}{20}
% by crazy default pmatrix supports only 10 cols :)


\usepackage{fontspec}
\usepackage{libertine}
\usepackage{polyglossia}

\setmainlanguage{russian}
\setotherlanguages{english}

% download "Linux Libertine" fonts:
% http://www.linuxlibertine.org/index.php?id=91&L=1
% \setmainfont{Linux Libertine O} % or Helvetica, Arial, Cambria
% why do we need \newfontfamily:
% http://tex.stackexchange.com/questions/91507/
% \newfontfamily{\cyrillicfonttt}{Linux Libertine O}

\AddEnumerateCounter{\asbuk}{\russian@alph}{щ} % для списков с русскими буквами
\setlist[enumerate, 2]{label=\asbuk*),ref=\asbuk*}

%% эконометрические сокращения
\DeclareMathOperator{\Cov}{\mathbb{C}ov}
\DeclareMathOperator{\plim}{plim}
\DeclareMathOperator{\Corr}{\mathbb{C}orr}
\DeclareMathOperator{\Var}{\mathbb{V}ar}
\DeclareMathOperator{\col}{col}
\DeclareMathOperator{\row}{row}
\DeclareMathOperator{\pCorr}{\mathrm{pCorr}}

\let\P\relax
\DeclareMathOperator{\P}{\mathbb{P}}

\let\H\relax
\DeclareMathOperator{\H}{\mathbb{H}}


\DeclareMathOperator{\E}{\mathbb{E}}
% \DeclareMathOperator{\tr}{trace}
\DeclareMathOperator{\card}{card}

\DeclareMathOperator{\Convex}{Convex}

\newcommand \cN{\mathcal{N}}
\newcommand \cF{\mathcal{F}}
\newcommand \cH{\mathcal{H}}

\newcommand \dN{\mathcal{N}}
\newcommand \dBin{\mathrm{Bin}}
\newcommand \dExpo{\mathrm{Expo}}


\newcommand \RR{\mathbb{R}}
\newcommand \NN{\mathbb{N}}





\begin{document}

\section*{Домашнее задание 12}

% pmi, fall 2024

Дедлайн: 2025-03-12, 23:59.

Оцениваемые задачи:

\begin{enumerate}

    
\item Правильный кубик подбрасывается один раз.
Рассмотрим $Y$ — индикатор того, выпала ли четная грань
и $Z$ — индикатор того, выпало ли число больше двух.
\begin{enumerate}
\item Выпишите явно сигма-алгебру $\sigma(Y\cdot Z)$.
\item Выпишите явно сигма-алгебру $\sigma(Y, Z)$.
\end{enumerate}


\item Рассмотрим минимальную сигма-алгебру $\cF$ на $\RR$, в которую входят все конечные подмножества числовой прямой. 
\begin{enumerate}
    \item Приведите два примера бесконечных подмножеств, входящих в $\cF$.
    \item Приведите два примера числовых подмножеств, не входящих в $\cF$.
\end{enumerate}


\end{enumerate}

Бесценные задачи в удовольствие:

\begin{enumerate}[resume]


\item Сколько существует сигма-алгебр на множестве $\Omega$ из четырёх элементов?

\item Будем обозначать количество элементов множества с помощью $\card A$.
Рассмотрим подмножества натуральных чисел, $A \subseteq \mathbb{N}$.
Определим для подмножества плотность Чезаро (Cesaro density),
\[
\gamma(A)=\lim_{n\to \infty}\frac{\card (A\cap \{1,2,3, \ldots,n\})}{n}
\]
в тех случаях, когда этот предел существует.

Плотность Чезаро показывает, какую «долю» от всех натуральных чисел составляет указанное подмножество.
Обозначим с помощью $\cH$ все подмножества, имеющие плотность Чезаро.

Является ли набор $\cH$ сигма-алгеброй?

\item Случайные величины $(X_n)$ независимы и равновероятно равны $\pm 1$.
Накопленную сумму обозначим $S_n = X_1 + X_2 + \dots + X_n$.

\begin{enumerate}
    \item Сколько событий сигма-алгебрах $\sigma(S_5)$? $\sigma(X_5)$?
    \item Как связаны между собой сигма-алгебры $\sigma(X_1, X_2, \dots, X_{100})$ и $\sigma(S_1, S_2, \dots, S_{100})$?
\end{enumerate}

\item В лесу есть три вида грибов: рыжики, лисички и мухоморы. 
Попадаются они равновероятно и независимо друг от друга. 
Маша нашла 100 грибов. 
Пусть $R$ — количество рыжиков, $L$ — количество лисичек, а $M$ — количество мухоморов среди найденных грибов.
\begin{enumerate}
\item Сколько элементов $\sigma(R - M)$?
\item Сколько элементов $\sigma(R, M)$?
\item Измерима ли $L$ относительно $\sigma(R, M)$?
\item Измерима ли $L$ относительно $\sigma(R - M)$?
\end{enumerate}


\end{enumerate}


\end{document}
