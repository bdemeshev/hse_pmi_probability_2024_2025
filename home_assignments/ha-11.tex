\documentclass[12pt]{article}

% \usepackage{physics}

\usepackage{hyperref}
\hypersetup{
    colorlinks=true,
    linkcolor=blue,
    filecolor=magenta,      
    urlcolor=cyan,
    pdftitle={Overleaf Example},
    pdfpagemode=FullScreen,
    }

\usepackage{tikzducks}

\usepackage{tikz} % картинки в tikz
\usepackage{microtype} % свешивание пунктуации

\usepackage{array} % для столбцов фиксированной ширины

\usepackage{indentfirst} % отступ в первом параграфе

\usepackage{sectsty} % для центрирования названий частей
\allsectionsfont{\centering}

\usepackage{amsmath, amsfonts, amssymb} % куча стандартных математических плюшек

\usepackage{comment}

\usepackage[top=2cm, left=1.2cm, right=1.2cm, bottom=2cm]{geometry} % размер текста на странице

\usepackage{lastpage} % чтобы узнать номер последней страницы

\usepackage{enumitem} % дополнительные плюшки для списков
%  например \begin{enumerate}[resume] позволяет продолжить нумерацию в новом списке
\usepackage{caption}

\usepackage{url} % to use \url{link to web}


\newcommand{\smallduck}{\begin{tikzpicture}[scale=0.3]
    \duck[
        cape=black,
        hat=black,
        mask=black
    ]
    \end{tikzpicture}}

\usepackage{fancyhdr} % весёлые колонтитулы
\pagestyle{fancy}
\lhead{}
\chead{}
\rhead{Домашние задания для самураев}
\lfoot{}
\cfoot{}
\rfoot{}

\renewcommand{\headrulewidth}{0.4pt}
\renewcommand{\footrulewidth}{0.4pt}

\usepackage{tcolorbox} % рамочки!

\usepackage{todonotes} % для вставки в документ заметок о том, что осталось сделать
% \todo{Здесь надо коэффициенты исправить}
% \missingfigure{Здесь будет Последний день Помпеи}
% \listoftodos - печатает все поставленные \todo'шки


% более красивые таблицы
\usepackage{booktabs}
% заповеди из докупентации:
% 1. Не используйте вертикальные линни
% 2. Не используйте двойные линии
% 3. Единицы измерения - в шапку таблицы
% 4. Не сокращайте .1 вместо 0.1
% 5. Повторяющееся значение повторяйте, а не говорите "то же"


\setcounter{MaxMatrixCols}{20}
% by crazy default pmatrix supports only 10 cols :)


\usepackage{fontspec}
\usepackage{libertine}
\usepackage{polyglossia}

\setmainlanguage{russian}
\setotherlanguages{english}

% download "Linux Libertine" fonts:
% http://www.linuxlibertine.org/index.php?id=91&L=1
% \setmainfont{Linux Libertine O} % or Helvetica, Arial, Cambria
% why do we need \newfontfamily:
% http://tex.stackexchange.com/questions/91507/
% \newfontfamily{\cyrillicfonttt}{Linux Libertine O}

\AddEnumerateCounter{\asbuk}{\russian@alph}{щ} % для списков с русскими буквами
\setlist[enumerate, 2]{label=\asbuk*),ref=\asbuk*}

%% эконометрические сокращения
\DeclareMathOperator{\Cov}{\mathbb{C}ov}
\DeclareMathOperator{\plim}{plim}
\DeclareMathOperator{\Corr}{\mathbb{C}orr}
\DeclareMathOperator{\Var}{\mathbb{V}ar}
\DeclareMathOperator{\col}{col}
\DeclareMathOperator{\row}{row}
\DeclareMathOperator{\pCorr}{\mathrm{pCorr}}

\let\P\relax
\DeclareMathOperator{\P}{\mathbb{P}}

\let\H\relax
\DeclareMathOperator{\H}{\mathbb{H}}


\DeclareMathOperator{\E}{\mathbb{E}}
% \DeclareMathOperator{\tr}{trace}
\DeclareMathOperator{\card}{card}

\DeclareMathOperator{\Convex}{Convex}

\newcommand \cN{\mathcal{N}}
\newcommand \dN{\mathcal{N}}
\newcommand \dBin{\mathrm{Bin}}
\newcommand \dExpo{\mathrm{Expo}}


\newcommand \RR{\mathbb{R}}
\newcommand \NN{\mathbb{N}}





\begin{document}

\section*{Домашнее задание 11}

% pmi, fall 2024

Дедлайн: 2025-02-20, 23:59.

Оцениваемые задачи:

\begin{enumerate}

    
\item Рассмотрим три характеристических функции:
\begin{itemize}
    \item $\phi_X(t) = \exp(6it - 10t^2)$;
    \item $\phi_Y(t) = \exp(-1 + 5it - 3t^2 + \cos(t) + i \sin (t))$;
    \item $\phi_Z(t) = 0.2 \exp(-t^2) + 0.5 / (2 - \exp(it)) + 0.3 \exp(2025it)$.
\end{itemize}
\begin{enumerate}
    \item Укажите, как можно сгенерировать значения соответствующей случайной величины, используя известные классические законы распределения (биномиальное, Пуассона, экспоненциальное и так далее).
    \item Укажите математическое ожидание случайной величины. 
\end{enumerate}

\item Величины $X_n$ независимы и имеют распределение Пуассона с интенсивностью $\lambda_n = n$, а $Y_n = (X_n - \E(X_n)) / \sqrt{\Var(X_n)}$.
\begin{enumerate}
    \item Найдите характеристическую функцию $X_n$.
    \item Найдите характеристическую функцию $Y_n$.
    \item Докажите, что характеристические функции $Y_n$ сходятся к стандартной нормальной. 
\end{enumerate}


\end{enumerate}

Бесценные задачи в удовольствие:

\begin{enumerate}[resume]

\item Вася переписал с доски к себе в тетрадь характеристическую функцию некоторой случайной величины:
\[
\phi_X(t) = 3/4 + 2it - t^2 + o(t^2). 
\]
Сколько минимум ошибок сделал Вася? Укажите эти ошибки. 


\item Характеристическая функция случайной величины $X$ равна $\phi_X(t) = \cos^3 t$.

\begin{enumerate}
    \item Найдите ожидание и дисперсию величины $X$.
    \item Найдите закон распределения величины $X$.
\end{enumerate}

%Найдите характеристическую функцию для распределения Коши с функцией плотности $f(x) = 1/(\pi^2 (1 + x^2))$.

Подсказка: выразите косинус через экспоненты с помощью формулы Эйлера.


\item Величины $X$, $Y$ и $Z$ независимы. 
Величина $X$ имеет нормальное $\cN(4; 10)$ распределение, $Y$ — биномиальное $\dBin(2, 0.2)$ распределение, 
а $Z$ — экспоненциальное $\dExpo(3)$.

Найдите характеристическую функцию случайной величины $S = 2XY + 3YZ + 7$.


\item Рассмотрим множество случайных величин с характеристической функцией вида $\phi(t) = 1/(1 - ita)^b$, где $a$ и $b$ — это произвольные положительные параметры.
\begin{enumerate}
    \item Найдите математическое ожидание и дисперсию как функции от $a$ и $b$.
    \item Верно ли, что если взять две независимые случайные величины из данного класса c $a = 52$ и сложить их, то снова получится случайная величина из данного класса с $a = 52$?
\end{enumerate}


\end{enumerate}


\end{document}
