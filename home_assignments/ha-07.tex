\documentclass[12pt]{article}

% \usepackage{physics}

\usepackage{hyperref}
\hypersetup{
    colorlinks=true,
    linkcolor=blue,
    filecolor=magenta,      
    urlcolor=cyan,
    pdftitle={Overleaf Example},
    pdfpagemode=FullScreen,
    }

\usepackage{tikzducks}

\usepackage{tikz} % картинки в tikz
\usepackage{microtype} % свешивание пунктуации

\usepackage{array} % для столбцов фиксированной ширины

\usepackage{indentfirst} % отступ в первом параграфе

\usepackage{sectsty} % для центрирования названий частей
\allsectionsfont{\centering}

\usepackage{amsmath, amsfonts, amssymb} % куча стандартных математических плюшек

\usepackage{comment}

\usepackage[top=2cm, left=1.2cm, right=1.2cm, bottom=2cm]{geometry} % размер текста на странице

\usepackage{lastpage} % чтобы узнать номер последней страницы

\usepackage{enumitem} % дополнительные плюшки для списков
%  например \begin{enumerate}[resume] позволяет продолжить нумерацию в новом списке
\usepackage{caption}

\usepackage{url} % to use \url{link to web}


\newcommand{\smallduck}{\begin{tikzpicture}[scale=0.3]
    \duck[
        cape=black,
        hat=black,
        mask=black
    ]
    \end{tikzpicture}}

\usepackage{fancyhdr} % весёлые колонтитулы
\pagestyle{fancy}
\lhead{}
\chead{}
\rhead{Домашние задания для самураев}
\lfoot{}
\cfoot{}
\rfoot{}

\renewcommand{\headrulewidth}{0.4pt}
\renewcommand{\footrulewidth}{0.4pt}

\usepackage{tcolorbox} % рамочки!

\usepackage{todonotes} % для вставки в документ заметок о том, что осталось сделать
% \todo{Здесь надо коэффициенты исправить}
% \missingfigure{Здесь будет Последний день Помпеи}
% \listoftodos - печатает все поставленные \todo'шки


% более красивые таблицы
\usepackage{booktabs}
% заповеди из докупентации:
% 1. Не используйте вертикальные линни
% 2. Не используйте двойные линии
% 3. Единицы измерения - в шапку таблицы
% 4. Не сокращайте .1 вместо 0.1
% 5. Повторяющееся значение повторяйте, а не говорите "то же"


\setcounter{MaxMatrixCols}{20}
% by crazy default pmatrix supports only 10 cols :)


\usepackage{fontspec}
\usepackage{libertine}
\usepackage{polyglossia}

\setmainlanguage{russian}
\setotherlanguages{english}

% download "Linux Libertine" fonts:
% http://www.linuxlibertine.org/index.php?id=91&L=1
% \setmainfont{Linux Libertine O} % or Helvetica, Arial, Cambria
% why do we need \newfontfamily:
% http://tex.stackexchange.com/questions/91507/
% \newfontfamily{\cyrillicfonttt}{Linux Libertine O}

\AddEnumerateCounter{\asbuk}{\russian@alph}{щ} % для списков с русскими буквами
\setlist[enumerate, 2]{label=\asbuk*),ref=\asbuk*}

%% эконометрические сокращения
\DeclareMathOperator{\Cov}{\mathbb{C}ov}
\DeclareMathOperator{\Corr}{\mathbb{C}orr}
\DeclareMathOperator{\Var}{\mathbb{V}ar}
\DeclareMathOperator{\col}{col}
\DeclareMathOperator{\row}{row}
\DeclareMathOperator{\pCorr}{\mathrm{pCorr}}

\let\P\relax
\DeclareMathOperator{\P}{\mathbb{P}}

\let\H\relax
\DeclareMathOperator{\H}{\mathbb{H}}


\DeclareMathOperator{\E}{\mathbb{E}}
% \DeclareMathOperator{\tr}{trace}
\DeclareMathOperator{\card}{card}

\DeclareMathOperator{\Convex}{Convex}

\newcommand \cN{\mathcal{N}}
\newcommand \dN{\mathcal{N}}
\newcommand \dBin{\mathrm{Bin}}


\newcommand \RR{\mathbb{R}}
\newcommand \NN{\mathbb{N}}





\begin{document}

\section*{Домашнее задание 7}

Дедлайн: 2024-11-01.

Здесь $\H(Y \mid X)$ — это условная энтропия, а $\H(X, Y)$ — совместная энтропия. 
Будьте осторожны, некоторые авторы используют обозначение $\H(X, Y)$ для кросс-энтропии. 


\begin{enumerate}
\item Распределение вектора $(X, Y)$ задано таблицей

\begin{center}
    \begin{tabular}{lccc}
    	\toprule
    	    & $Y = 1$  & $Y = 2$  & $Y = 3$ \\
        \midrule
    	$X = 0$ & $0.2$  & $0.2$  & $0.1$ \\
        $X = 1$ & $0.5$  &  $0$   & $0$ \\
      \bottomrule
    \end{tabular}
\end{center}

\begin{enumerate}
    \item Найдите энтропии $\H(X)$, $\H(Y)$, $\H(X, Y)$.
    \item Найдите $\H(Y \mid X)$.
    \item Какое максимальное значение может принимать условная энтропия $\H(Y \mid X)$, 
    если $X$ принимает два значения, а $Y$ — три?
\end{enumerate}

\item Рассмотрим равномерное распределение на отрезке $[0; 1]$.

\begin{enumerate}
    \item Найдите энтропию равномерного распределения на отрезке $[0; 1]$.
    \item Докажите, что равномерное распределение имеет максимальную энтропию среди всех распределений на отрезке $[0; 1]$,
    имеющих функцию плотности. 
\end{enumerate}

Рассмотим распределение с функцией плотности $f(x) = \exp(-x^2/2)/\sqrt{2\pi}$ на числовой прямой. 
Кстати, оно называется \emph{стандартным нормальным}. 

\begin{enumerate}[resume]
    \item Найдите математическое ожидание и дисперсию данного распределения. 
    \item Найдите энтропию стандартного нормального распределения.     
    \item Докажите, что стандартное нормальное распределение имеет максимальную энтропию среди всех распределений с функцией плотности с нулевым математическим ожиданием и единичной дисперсией.
\end{enumerate}

Подсказка: можно без доказательства пользоваться тем, что $\int_{-\infty}^{+\infty} f(x) dx = 1$.

\item Для дискретных величин $X$ и $Y$ докажите или опровергните утверждения:
\begin{enumerate}
    \item $\H(X) + \H(Y \mid X) = \H(X, Y)$;
    \item $\H(X, Y) \geq \H(X)$;
    \item $\H(X^2) = \H(X)$;
\end{enumerate}


\end{enumerate}




\end{document}
