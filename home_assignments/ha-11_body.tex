\section*{Домашнее задание 11}

% pmi, fall 2024

Дедлайн: 2025-02-20, 23:59.

Оцениваемые задачи:

\begin{enumerate}

    
\item Рассмотрим три характеристических функции:
\begin{itemize}
    \item $\phi_X(t) = \exp(6it - 10t^2)$;
    \item $\phi_Y(t) = \exp(-1 + 5it - 3t^2 + \cos(t) + i \sin (t))$;
    \item $\phi_Z(t) = 0.2 \exp(-t^2) + 0.5 / (2 - \exp(it)) + 0.3 \exp(2025it)$.
\end{itemize}
\begin{enumerate}
    \item Укажите, как можно сгенерировать значения соответствующей случайной величины, используя известные классические законы распределения (биномиальное, Пуассона, экспоненциальное и так далее).
    \item Укажите математическое ожидание случайной величины. 
\end{enumerate}

\item Величины $X_n$ независимы и имеют распределение Пуассона с интенсивностью $\lambda_n = n$, а $Y_n = (X_n - \E(X_n)) / \sqrt{\Var(X_n)}$.
\begin{enumerate}
    \item Найдите характеристическую функцию $X_n$.
    \item Найдите характеристическую функцию $Y_n$.
    \item Докажите, что характеристические функции $Y_n$ сходятся к стандартной нормальной. 
\end{enumerate}


\end{enumerate}

Бесценные задачи в удовольствие:

\begin{enumerate}[resume]

\item Вася переписал с доски к себе в тетрадь характеристическую функцию некоторой случайной величины:
\[
\phi_X(t) = 3/4 + 2it - t^2 + o(t^2). 
\]
Сколько минимум ошибок сделал Вася? Укажите эти ошибки. 


\item Характеристическая функция случайной величины $X$ равна $\phi_X(t) = \cos^3 t$.

\begin{enumerate}
    \item Найдите ожидание и дисперсию величины $X$.
    \item Найдите закон распределения величины $X$.
\end{enumerate}

%Найдите характеристическую функцию для распределения Коши с функцией плотности $f(x) = 1/(\pi^2 (1 + x^2))$.

Подсказка: выразите косинус через экспоненты с помощью формулы Эйлера.


\item Величины $X$, $Y$ и $Z$ независимы. 
Величина $X$ имеет нормальное $\cN(4; 10)$ распределение, $Y$ — биномиальное $\dBin(2, 0.2)$ распределение, 
а $Z$ — экспоненциальное $\dExpo(3)$.

Найдите характеристическую функцию случайной величины $S = 2XY + 3YZ + 7$.


\item Рассмотрим множество случайных величин с характеристической функцией вида $\phi(t) = 1/(1 - ita)^b$, где $a$ и $b$ — это произвольные положительные параметры.
\begin{enumerate}
    \item Найдите математическое ожидание и дисперсию как функции от $a$ и $b$.
    \item Верно ли, что если взять две независимые случайные величины из данного класса c $a = 52$ и сложить их, то снова получится случайная величина из данного класса с $a = 52$?
\end{enumerate}


\end{enumerate}
