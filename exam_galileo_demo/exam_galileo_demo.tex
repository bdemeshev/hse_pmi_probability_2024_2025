% arara: xelatex
\documentclass[12pt]{article}

% \usepackage{physics}

\usepackage{hyperref}
\hypersetup{
    colorlinks=true,
    linkcolor=blue,
    filecolor=magenta,      
    urlcolor=cyan,
    pdftitle={Overleaf Example},
    pdfpagemode=FullScreen,
    }
\urlstyle{same}

\usepackage{tikzducks}

\usepackage{tikz} % картинки в tikz
\usepackage{microtype} % свешивание пунктуации

\usepackage{array} % для столбцов фиксированной ширины

\usepackage{indentfirst} % отступ в первом параграфе

\usepackage{sectsty} % для центрирования названий частей
\allsectionsfont{\centering}

\usepackage{amsmath, amsfonts, amssymb} % куча стандартных математических плюшек

\usepackage{mathtools}
\usepackage{comment}

\usepackage[top=2cm, left=1.2cm, right=1.2cm, bottom=2cm]{geometry} % размер текста на странице

\usepackage{lastpage} % чтобы узнать номер последней страницы

\usepackage{enumitem} % дополнительные плюшки для списков
%  например \begin{enumerate}[resume] позволяет продолжить нумерацию в новом списке
\usepackage{caption}

\usepackage{url} % to use \url{link to web}


\newcommand{\smallduck}{\begin{tikzpicture}[scale=0.3]
    \duck[
        cape=black,
        hat=black,
        mask=black
    ]
    \end{tikzpicture}}

\usepackage{fancyhdr} % весёлые колонтитулы
\pagestyle{fancy}
\lhead{}
\chead{}
\rhead{Демо версии на февраль 2025}
\lfoot{}
\cfoot{}
\rfoot{}

\renewcommand{\headrulewidth}{0.4pt}
\renewcommand{\footrulewidth}{0.4pt}

\usepackage{tcolorbox} % рамочки!

\usepackage{todonotes} % для вставки в документ заметок о том, что осталось сделать
% \todo{Здесь надо коэффициенты исправить}
% \missingfigure{Здесь будет Последний день Помпеи}
% \listoftodos - печатает все поставленные \todo'шки


% более красивые таблицы
\usepackage{booktabs}
% заповеди из докупентации:
% 1. Не используйте вертикальные линни
% 2. Не используйте двойные линии
% 3. Единицы измерения - в шапку таблицы
% 4. Не сокращайте .1 вместо 0.1
% 5. Повторяющееся значение повторяйте, а не говорите "то же"


\setcounter{MaxMatrixCols}{20}
% by crazy default pmatrix supports only 10 cols :)


\usepackage{fontspec}
\usepackage{libertine}
\usepackage{polyglossia}

\setmainlanguage{russian}
\setotherlanguages{english}

% download "Linux Libertine" fonts:
% http://www.linuxlibertine.org/index.php?id=91&L=1
% \setmainfont{Linux Libertine O} % or Helvetica, Arial, Cambria
% why do we need \newfontfamily:
% http://tex.stackexchange.com/questions/91507/
% \newfontfamily{\cyrillicfonttt}{Linux Libertine O}

\AddEnumerateCounter{\asbuk}{\russian@alph}{щ} % для списков с русскими буквами
\setlist[enumerate, 2]{label=\asbuk*),ref=\asbuk*}

%% эконометрические сокращения
\DeclareMathOperator{\Cov}{\mathbb{C}ov}
\DeclareMathOperator{\Corr}{\mathbb{C}orr}
\DeclareMathOperator{\Var}{\mathbb{V}ar}
\DeclareMathOperator{\pCorr}{\mathrm{pCorr}}
\DeclareMathOperator{\col}{col}
\DeclareMathOperator{\row}{row}

\let\P\relax
\DeclareMathOperator{\P}{\mathbb{P}}

\DeclarePairedDelimiter{\abs}{\lvert}{\rvert}
\DeclarePairedDelimiter{\scalp}{\langle}{\rangle}

\let\H\relax
\DeclareMathOperator{\H}{\mathbb{H}}
\DeclareMathOperator{\plim}{plim}

\DeclareMathOperator{\E}{\mathbb{E}}
% \DeclareMathOperator{\tr}{trace}
\DeclareMathOperator{\card}{card}

\DeclareMathOperator{\Convex}{Convex}

\newcommand \cN{\mathcal{N}}
\newcommand \dN{\mathcal{N}}


\newcommand \RR{\mathbb{R}}
\newcommand \NN{\mathbb{N}}

\newcommand{\dBern}{\mathrm{Bern}}
\newcommand{\dBin}{\mathrm{Bin}}
\newcommand{\dGamma}{\mathrm{Gamma}}
\newcommand{\dBeta}{\mathrm{Beta}}



\begin{document}

\section*{Формат}

В работе будет 6 задач. 
Задачи имеют равный вес. 
Продолжительность работы 120 минут. 
Можно будет использовать в качестве разрешенной шпаргалки один лист А4 со всех шести его сторон.

\section*{Вариант «Птолемей»}
\begin{enumerate}
    \item % одномерное нормальное 
    Случайные величины $X$ и $Y$ заданы на одном множестве исходов $\Omega$ и величину $W = \max\{ X, Y \}$.
    \begin{enumerate}
        \item Докажите, что случайная величина $W$ измерима относительно сигма-алгебры $\sigma(X, Y)$.
        \item Приведите два примера нетривиальных событий (отличных от $\emptyset$ и $\Omega$), которые лежат в $\sigma(X, Y)$, но при этом не лежат в $\sigma(W)$.
        \item Верно ли, что $\sigma(W) \cup \sigma(X, Y)$ — сигма-алгебра?
    \end{enumerate}
    
    
    \item На числовой прямой $\RR$ заданы два набора подмножеств, $\mathcal{A} = \{\text{все интервалы вида } (a; b], \text{ где }a, b\in \RR \}$ и 
$\mathcal{I} = \{\text{все интервалы вида } [a; b), \text{ где }a, b\in \mathbb{Q} \}$.

Совпадают ли сигма-алгебры $\sigma(A)$ и $\sigma(B)$?

Если совпадают, то докажите их совпадение. 
Если не совпадают, то приведите два примера множеств, которыми эти сигма-алгебры отличаются. 

    \item Рассмотрим последовательность характеристических функций 
    \[
    \phi_n(x) = \exp(3it/n - 4(1 + 1/n^2)t^2)(0.3 \exp(it) + 0.7\exp(-it)).
    \]
    Обозначим с помощью $L$ — случайную величину, соответствующую пределу данной последовательности. 
    \begin{enumerate}
        \item Найдите предел последовательности $\phi_n$ при $n\to \infty$.
        \item Представьте $L$ в виде суммы двух величин с классическими законами распределения.
        \item Найдите $\E(L)$ и $\Var(L)$.
        \item Что можно утверждать про независимость случайных величин последовательности $(X_n)$ с характеристическими функциями $(\phi_n)$?
        % \item Приведите пример последовательности характеристических функций, у которой есть предел, но ни одна случайная величина пределу не соответствует. 
    \end{enumerate}

    \item Клавдий подкидывает правильную монетку один раз. 
    Если монетка выпадает орлом, то он складывает 50 независимых экспоненциально распределенных величин с интенсивностью $1$ каждая. 
    Если монетка выпадает решкой, то он складывает 100 независимых экспоненциально распределенных величин с интенсивностью $2$ каждая. 
    В результате Клавдий получает случайную величину $K$.

    \begin{enumerate}
        \item Найдите характеристическую функцию случайной величины $K$.
        \item Разложите полученную характеристическую функцию $\phi(t)$ в ряд Тейлора до $o(t^2)$.
        \item Какой вероятностный смысл несёт величина $\phi''(2025)$?
    \end{enumerate}

    \item Величины $(u_n)$ независимы и одинаково распределены с ожиданием $10$ и дисперсией $20$.
    Моменты $\E(u_n^3)$ и $\E(u_n^4)$ конечны.
    Определим $X_n = (u_1 + u_2 + \dots + u_n)^2 / n^2$.

    \begin{enumerate}
        \item Найдите $\lim \E(X_n)$.
        \item Найдите $\lim \Var(X_n)$.
        \item Найдите предел по вероятности $\plim X_n$.
    \end{enumerate}
    
    \item Величины $X_n$ распределены биномиально $\dBin(n, 1/2)$.
    \begin{enumerate}
        \item Найдите предел по распределению последовательности $(X_n - \E(X_n)) / \sqrt{\Var(X_n)}$.
        \item Найдите предел по вероятности последовательности $n\Var(X_n) / (nX_n  - X^2_n)$.
        \item Найдите предел по распределению последовательности $\sqrt{n} (X_n - \E(X_n)) / \sqrt{nX_n  - X^2_n}$.
    \end{enumerate}

    Уточнение: можно опираться на центральную предельную теорему и леммы Слуцкого. 

\end{enumerate}


\section*{Вариант «Коперник»}

\begin{enumerate}
    \item Величины $X_n$ имеют функцию плотности $f(x) = n x^{n-1}$ на отрезке $[0;1]$.
    \begin{enumerate}
        \item Найдите $\lim \E(S_n)$.
        \item Найдите $\lim \Var(S_n)$.
        \item К чему и в каких смыслах (по вероятности, почти наверное, по распределению, в $L^1$, в $L^2$) сходится последовательность $X_n$?
    \end{enumerate}

    \item Николаю нужно сложить $100$ независимых величин, равномерных на отрезке $[0;1]$.
    Однако каждую величину Николай забывает добавить в сумму с вероятностью $0.1$, независимо значения величин и от того, добавил ли он остальные величины. 
    В результате Николай получает случайную величину $N$.

    \begin{enumerate}
        \item Найдите характеристическую функцию случайной величины $N$.
        \item Разложите полученную характеристическую функцию $\phi(t)$ в ряд Тейлора до $o(t^2)$.
        \item Найдите математическое ожидание и дисперсию $N$.
    \end{enumerate}


    \item Величины $(u_n)$ независимы и одинаково распределены с ожиданием $10$ и дисперсией $20$.
    Определим $y_n = u_n + u_{n-1}$ и $S_n = y_1 + y_2 + \dots + y_n$.

    \begin{enumerate}
        \item Найдите $\E(S_n)$ и $\lim \E(S_n)$.
        \item Найдите $\Var(S_n)$ и $\lim \Var(S_n)$.
        \item Найдите предел по распределению последовательности $(S_n - \E(S_n))/\sqrt{n}$.
    \end{enumerate}

    \item Последовательность $(X_n)$ сходится по вероятности к величине $X$, а про случайную величину $Y$ ничего не известно. 
    \begin{enumerate}
        \item Вспомнив аддитивность вероятности, с обоснованием найдите предел $\lim \P(\abs{Y} \leq c_n)$, если $c_n \to \infty$. 
        \item С обоснованием найдите предел по вероятности последовательности $Y \cdot X_n$. 
    \end{enumerate}

    Уточнение: в пункте (б) можно опираться только на определение сходимости по вероятности. 

    \item Царь Кощей может в пределах своего благостояния каждый день утром закупать или продавать мем-койн YAGA. 
    Курс YAGA за каждые сутки независимо от других с вероятностью $0.8$ растёт в $2$ раза, а с вероятностью $0.2$ падает в $10$ раз. 
    
    Изначально у Кощея $S_0 = 100$ рублей, инфляция в рублях равна нулю\footnote{это сказочная задача!}.

    Определим долгосрочную дневную процентную ставку $r$ условием $\plim S_n / (1 + r)^n = S_0$.

    \begin{enumerate}
        \item Чему равна долгосрочная дневная процентная ставка, если Кощей держит все свои деньги в YAGA?
        \item Как выглядит стратегия Кощея, максимизирующая долгосрочную дневную процентную ставку?
    \end{enumerate}


    Подсказка: \url{https://en.wikipedia.org/wiki/Kelly_criterion}

    \item Царевна Несмеяна любит читать несмешные книжки. 
    Всего у неё 8 книг, $k$-я по счёту книга оказывается несмешной с вероятностью $1/k$.
    Цель Несмеяны — максимизировать вероятность прочесть все несмешные книги за наименьшее число попыток. 

    Как выглядит оптимальная стратегия Несмеяны?

    Подсказка: \url{https://en.wikipedia.org/wiki/Odds_algorithm}

\end{enumerate}


\end{document}

%Из определения условного ожидания, $\E(X \mid A) = \E(X \cdot I_A) / \P(A)$, 
%легко получаются определения условной дисперсии, $\Var(X \mid A) = \E(X^2 \mid A) - (\E(X \mid A))^2$,
%условной ковариации $\Cov(X, Y \mid A) = \E(XY \mid A) - \E(X \mid A) \E(Y \mid A)$ и даже корреляции, 
%$\Corr(X, Y \mid A) = \Cov(X, Y \mid A) / \sqrt{\Var(X \mid A)\Var(Y \mid A)}$.

% здесь проектируемая часть



\end{document}

