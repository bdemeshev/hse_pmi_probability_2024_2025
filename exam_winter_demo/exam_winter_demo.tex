% arara: xelatex
\documentclass[12pt]{article}

% \usepackage{physics}

\usepackage{hyperref}
\hypersetup{
    colorlinks=true,
    linkcolor=blue,
    filecolor=magenta,      
    urlcolor=cyan,
    pdftitle={Overleaf Example},
    pdfpagemode=FullScreen,
    }

\usepackage{tikzducks}

\usepackage{tikz} % картинки в tikz
\usepackage{microtype} % свешивание пунктуации

\usepackage{array} % для столбцов фиксированной ширины

\usepackage{indentfirst} % отступ в первом параграфе

\usepackage{sectsty} % для центрирования названий частей
\allsectionsfont{\centering}

\usepackage{amsmath, amsfonts, amssymb} % куча стандартных математических плюшек

\usepackage{mathtools}
\usepackage{comment}

\usepackage[top=2cm, left=1.2cm, right=1.2cm, bottom=2cm]{geometry} % размер текста на странице

\usepackage{lastpage} % чтобы узнать номер последней страницы

\usepackage{enumitem} % дополнительные плюшки для списков
%  например \begin{enumerate}[resume] позволяет продолжить нумерацию в новом списке
\usepackage{caption}

\usepackage{url} % to use \url{link to web}


\newcommand{\smallduck}{\begin{tikzpicture}[scale=0.3]
    \duck[
        cape=black,
        hat=black,
        mask=black
    ]
    \end{tikzpicture}}

\usepackage{fancyhdr} % весёлые колонтитулы
\pagestyle{fancy}
\lhead{}
\chead{}
\rhead{Демо версии на декабрь 2024}
\lfoot{}
\cfoot{}
\rfoot{}

\renewcommand{\headrulewidth}{0.4pt}
\renewcommand{\footrulewidth}{0.4pt}

\usepackage{tcolorbox} % рамочки!

\usepackage{todonotes} % для вставки в документ заметок о том, что осталось сделать
% \todo{Здесь надо коэффициенты исправить}
% \missingfigure{Здесь будет Последний день Помпеи}
% \listoftodos - печатает все поставленные \todo'шки


% более красивые таблицы
\usepackage{booktabs}
% заповеди из докупентации:
% 1. Не используйте вертикальные линни
% 2. Не используйте двойные линии
% 3. Единицы измерения - в шапку таблицы
% 4. Не сокращайте .1 вместо 0.1
% 5. Повторяющееся значение повторяйте, а не говорите "то же"


\setcounter{MaxMatrixCols}{20}
% by crazy default pmatrix supports only 10 cols :)


\usepackage{fontspec}
\usepackage{libertine}
\usepackage{polyglossia}

\setmainlanguage{russian}
\setotherlanguages{english}

% download "Linux Libertine" fonts:
% http://www.linuxlibertine.org/index.php?id=91&L=1
% \setmainfont{Linux Libertine O} % or Helvetica, Arial, Cambria
% why do we need \newfontfamily:
% http://tex.stackexchange.com/questions/91507/
% \newfontfamily{\cyrillicfonttt}{Linux Libertine O}

\AddEnumerateCounter{\asbuk}{\russian@alph}{щ} % для списков с русскими буквами
\setlist[enumerate, 2]{label=\asbuk*),ref=\asbuk*}

%% эконометрические сокращения
\DeclareMathOperator{\Cov}{\mathbb{C}ov}
\DeclareMathOperator{\Corr}{\mathbb{C}orr}
\DeclareMathOperator{\Var}{\mathbb{V}ar}
\DeclareMathOperator{\pCorr}{\mathrm{pCorr}}
\DeclareMathOperator{\col}{col}
\DeclareMathOperator{\row}{row}

\let\P\relax
\DeclareMathOperator{\P}{\mathbb{P}}

\DeclarePairedDelimiter{\abs}{\lvert}{\rvert}
\DeclarePairedDelimiter{\scalp}{\langle}{\rangle}

\let\H\relax
\DeclareMathOperator{\H}{\mathbb{H}}


\DeclareMathOperator{\E}{\mathbb{E}}
% \DeclareMathOperator{\tr}{trace}
\DeclareMathOperator{\card}{card}

\DeclareMathOperator{\Convex}{Convex}

\newcommand \cN{\mathcal{N}}
\newcommand \dN{\mathcal{N}}


\newcommand \RR{\mathbb{R}}
\newcommand \NN{\mathbb{N}}

\newcommand{\dBern}{\mathrm{Bern}}
\newcommand{\dBin}{\mathrm{Bin}}
\newcommand{\dGamma}{\mathrm{Gamma}}
\newcommand{\dBeta}{\mathrm{Beta}}



\begin{document}

\section*{Формат}

В работе будет 6 задач. 
Задачи имеют равный вес. 
Продолжительность работы 120 минут. 
На декабрьской письменной работе можно будет использовать в качестве разрешенноё шпаргалки один лист А4 со всех шести его сторон.
В задачах про нормальное распределение нужно уметь как воспользоваться таблицей, так и записать ответ с помощью функции распределения $F()$ для нормальной стандартной случайной величины.

\section*{Демо «Колотун»}
\begin{enumerate}
    \item % одномерное нормальное 
    Случайные величины $X_1$ и $X_2$ независимы и имеют нормальное стандартное распределение $\cN(0;1)$.
    \begin{enumerate}
        \item Найдите вероятность $\P(X_1^2 + X_2^2 \leq t)$.
        \item Какое распределение имеет случайная величина $S = X_1^2 + X_2^2$?
    \end{enumerate}
    
    \item Вектор $Y$ имеет совместное нормальное распределение. 
    \[
    Y \sim \cN\left( \begin{pmatrix}
        1 \\
        2 \\
        5 \\
    \end{pmatrix}, 
    \begin{pmatrix}
        10 & 2 & 1 \\
         & 20 & -1 \\
         & & 30 \\
    \end{pmatrix}
    \right).
    \]
\begin{enumerate}
    \item Найдите $\E(Y_1 - 5Y_2)$, $\Var(Y_1 - 5Y_2)$, $\P(Y_1 - 5Y_2 > 0)$.
    \item Найдите $\Cov(Y_1 Y_2, Y_2 Y_3)$.
    \item Найдите $\P(Y_1 > 3 \mid Y_2 = 5)$.
\end{enumerate}    


    \item Вектор $(X, Y)$ имеет совместную функцию плотности 
    \[
    f(x, y) = \begin{cases}
        x + 5y^9, \text{ если } x, y \in [0;1], \\
        0, \text{ иначе.}
    \end{cases}
    \]
    \begin{enumerate}
        \item Найдите совместную функцию плотности вектора $(R = X - Y^3, S = X + Y^3)$.
        \item Найдите условную функцию плотности $f_{Y\mid X}(y \mid x)$.
        \item Найдите условные $\E(Y\mid X = x)$ и $\Var(Y \mid X = x)$.
    \end{enumerate}

    \item Крипто-портфель инвестора Кота Базилио состоит из двух альт-койнов с вектором доходностей $R = (R_1, R_2)$ (в долях от единицы).
    \[
    \E(R) = \begin{pmatrix}
        0.1 \\
        0.5 \\
    \end{pmatrix}, 
    \Var(R) = \begin{pmatrix}
        1 & -0.5 \\
        -0.5 & 10 \\
    \end{pmatrix}.
    \]
    Базилио может включить в свой портфель альт-койны с вектором весов $w = (w_1, 1 - w_1)$, где $w_1 \in [0;1]$.
    Доходность портфеля считаем как скалярное произведение $R_P = \scalp{w, R}$.
    
    \begin{enumerate}
        \item Какой портфель минимизирует дисперсию $\Var(R_P) \to \min_w$?
        \item Какая будет доходность у портфеля с минимальной дисперсией?
    \end{enumerate}

    \item Спамеры звонят мне согласно пуассоновскому потоку с интенсивностью $\lambda = 5$ звонков в неделю. 
    \begin{enumerate}
        \item Какова вероятность того, что за один день поступит не более одного звонка?
        \item Найдите функцию плотности времени, которое пройдёт от начала наблюдения до третьего звонка. 
        \item Найдите математическое ожидание и дисперсию числа звонков за три дня. 
    \end{enumerate}

    \item Случайные величины $X_1$, \dots, $X_{10}$ независимы и имеют функцию плотности $2x$ на отрезке $[0;1]$.
    Упорядочим их по возрастанию и рассмотрим порядковые статистики $X_{(1)} \leq X_{(2)} \leq \dots \leq X_{(n)}$.
    \begin{enumerate}
        \item Найдите функцию плотности минимальной порядковой статистики $X_{(1)}$.
        \item Найдите функцию плотности величины $X_{(3)}$.
        \item Найдите совместную функцию плотности пары $(X_{(3)}, X_{(7)})$.
    \end{enumerate}
\end{enumerate}


\section*{Вариант «Пурга»}

\begin{enumerate}
    \item Машенькина оценка за контрольную $X$ распределена равномерно на отрезке $[0, 1]$. 
    Вовочка списывает у Маши, но с ошибками, поэтому его оценка $Y$ за контрольную условно распределена равномерно на отрезке
    $[0, X]$. 

    \begin{enumerate}
        \item Выпишите условную функцию плотности $f_{Y\mid X}(y \mid x)$.
        \item Восстановите совместную функцию плотности $f(x, y)$.
        \item Найдите ковариацию $\Cov(X, Y)$.
    \end{enumerate}

    \item  Монетка выпадает орлом с вероятностью $1/2$. 
    Эксперимент состоит из двух этапов. 
    На первом этапе монетку подкидывают 100 раз и записывают число орлов, $R$. 
    На втором этапе монетку подбрасывают до тех пор пока не выпадет столько орлов, сколько выпало на первом этапе. 
    Обозначим число подбрасываний монетки на втором этапе буквой $S$. 
   
    Найдите ожидание $\E(S \mid R = r)$ и дисперсию $\Var(S \mid R = r)$.

    \item В пятницу 13 сентября 2019 года в Атланте перевернулся грузовик с 216 тысячями игральных кубиков. 
    К счастью, никто не пострадал. 

    Предположим, что все кубики выпали на дорогу. 
    \begin{enumerate}
      \item Какова вероятность того, что в сумме выпало больше 740000?
      \item Найдите такое число $a$, чтобы вероятность того, что выпала сумма меньше $a$, равнялась $0.777$.
    \end{enumerate}

    Предположим, что часть кубиков осталась в грузовике.
    \begin{enumerate}[resume]
      \item Какая часть кубиков выпала на дорогу, если 
      вероятность того, что сумма на кубиках, выпавших на дорогу,
      больше суммы на кубиках, оставшихся в грузовике, равна $2/3$?
    \end{enumerate}
  
    \item Величины $X_1$, $X_2$, \dots, $X_n$ независимы и экспоненциально распределены с интенсивностью $\lambda = 1$,
    а $S = X_1 + \dots + X_{1000}$ и $R = X_{501} + \dots + X_{1500}$.

    \begin{enumerate}
        \item Примерно оцените вероятность $\P(S > 1050)$.
        \item Примерно найдите условное ожидание $\E(S \mid S > 1000)$.
        \item Какое примерно распределение имеет вектор $(S, R)$?
    \end{enumerate}

    \item Есть пара независимых случайных величин, $R \sim \dBeta(10, 20)$ и $S \sim \dGamma(30, \lambda = 5)$.
    \begin{enumerate}
        \item Найдите с доказательством моду $R$ и моду $S$. 
        \item Найдите (можно без доказательства) $\E(R)$ и $\E(S)$.
        \item Найдите закон распределения $W = S\cdot R$ и закон распределения $Q = S \cdot (1 - R)$.
    \end{enumerate}

    \item Величины $X_1$, $X_2$, \dots{ }, $X_n$ независимы и равномерно распределены на отрезке $[0;1]$,
    а $Y_n$ — наименьшее из этих $n$ чисел.
    \begin{enumerate}
        \item К чему сходится последовательность $Y_n$ по распределению?
        \item К чему сходится последовательность $R_n = n\cdot Y_n$ по распределению?
    \end{enumerate}


\end{enumerate}


\end{document}

%Из определения условного ожидания, $\E(X \mid A) = \E(X \cdot I_A) / \P(A)$, 
%легко получаются определения условной дисперсии, $\Var(X \mid A) = \E(X^2 \mid A) - (\E(X \mid A))^2$,
%условной ковариации $\Cov(X, Y \mid A) = \E(XY \mid A) - \E(X \mid A) \E(Y \mid A)$ и даже корреляции, 
%$\Corr(X, Y \mid A) = \Cov(X, Y \mid A) / \sqrt{\Var(X \mid A)\Var(Y \mid A)}$.

% здесь проектируемая часть



\end{document}

