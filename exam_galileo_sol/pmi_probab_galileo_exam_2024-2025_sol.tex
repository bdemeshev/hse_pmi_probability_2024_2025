% arara: xelatex
\documentclass[12pt]{article}

% \usepackage{physics}

\usepackage{hyperref}
\hypersetup{
    colorlinks=true,
    linkcolor=blue,
    filecolor=magenta,      
    urlcolor=cyan,
    pdftitle={Overleaf Example},
    pdfpagemode=FullScreen,
    }

\usepackage{tikzducks}

\usepackage{tikz} % картинки в tikz
\usetikzlibrary{shapes, arrows, positioning}
\usepackage{microtype} % свешивание пунктуации

\usepackage{array} % для столбцов фиксированной ширины

\usepackage{indentfirst} % отступ в первом параграфе

\usepackage{sectsty} % для центрирования названий частей
\allsectionsfont{\centering}

\usepackage{amsmath, amsfonts, amssymb} % куча стандартных математических плюшек

\usepackage{comment}

\usepackage[top=2cm, left=1.2cm, right=1.2cm, bottom=2cm]{geometry} % размер текста на странице

\usepackage{lastpage} % чтобы узнать номер последней страницы

\usepackage{enumitem} % дополнительные плюшки для списков
%  например \begin{enumerate}[resume] позволяет продолжить нумерацию в новом списке
\usepackage{caption}

\usepackage{url} % to use \url{link to web}


\newcommand{\smallduck}{\begin{tikzpicture}[scale=0.3]
    \duck[
        cape=black,
        hat=black,
        mask=black
    ]
    \end{tikzpicture}}

\usepackage{fancyhdr} % весёлые колонтитулы
\pagestyle{fancy}
\lhead{Теория вероятностей для самураев}
\chead{}
\rhead{Разбалловка}
\lfoot{}
%\lfoot{$F(0.6) = 0.73$, $F(0.4) = 0.66$, $F(-0.25) = 0.4$, $F(1.03) = 0.85$, $F(0.05) = 0.52$ }
\cfoot{}
\rfoot{}




\renewcommand{\headrulewidth}{0.4pt}
\renewcommand{\footrulewidth}{0.4pt}

\usepackage{tcolorbox} % рамочки!

\usepackage{todonotes} % для вставки в документ заметок о том, что осталось сделать
% \todo{Здесь надо коэффициенты исправить}
% \missingfigure{Здесь будет Последний день Помпеи}
% \listoftodos - печатает все поставленные \todo'шки


% более красивые таблицы
\usepackage{booktabs}
% заповеди из докупентации:
% 1. Не используйте вертикальные линни
% 2. Не используйте двойные линии
% 3. Единицы измерения - в шапку таблицы
% 4. Не сокращайте .1 вместо 0.1
% 5. Повторяющееся значение повторяйте, а не говорите "то же"


\setcounter{MaxMatrixCols}{20}
% by crazy default pmatrix supports only 10 cols :)


\usepackage{fontspec}
\usepackage{libertine}
\usepackage{polyglossia}

\setmainlanguage{russian}
\setotherlanguages{english}

% download "Linux Libertine" fonts:
% http://www.linuxlibertine.org/index.php?id=91&L=1
% \setmainfont{Linux Libertine O} % or Helvetica, Arial, Cambria
% why do we need \newfontfamily:
% http://tex.stackexchange.com/questions/91507/
% \newfontfamily{\cyrillicfonttt}{Linux Libertine O}

\AddEnumerateCounter{\asbuk}{\russian@alph}{щ} % для списков с русскими буквами
\setlist[enumerate, 2]{label=\asbuk*),ref=\asbuk*}

%% эконометрические сокращения
\DeclareMathOperator{\Cov}{\mathbb{C}ov}
\DeclareMathOperator{\Corr}{\mathbb{C}orr}
\DeclareMathOperator{\Var}{\mathbb{V}ar}
\DeclareMathOperator{\col}{col}
\DeclareMathOperator{\row}{row}

\DeclareMathOperator{\rank}{rank}

\let\P\relax
\DeclareMathOperator{\P}{\mathbb{P}}

\let\H\relax
\DeclareMathOperator{\H}{\mathbb{H}}


\DeclareMathOperator{\E}{\mathbb{E}}
% \DeclareMathOperator{\tr}{trace}
\DeclareMathOperator{\card}{card}

\DeclareMathOperator{\Convex}{Convex}
\DeclareMathOperator{\plim}{plim}

\newcommand{\cN}{\mathcal{N}}
\newcommand{\cF}{\mathcal{F}}


\newcommand{\SST}{\text{SST}}
\newcommand{\SSR}{\text{SS}^{\text{res}}}

\newcommand{\RR}{\mathbb{R}}
\newcommand{\NN}{\mathbb{N}}
\newcommand{\hb}{\hat{\beta}}
\newcommand{\dPois}{\mathrm{Pois}}
\newcommand{\dBin}{\mathrm{Bin}}

\usepackage{mathtools}
\DeclarePairedDelimiter{\norm}{\lVert}{\rVert}
\DeclarePairedDelimiter{\abs}{\lvert}{\rvert}
\DeclarePairedDelimiter{\scalp}{\langle}{\rangle}
\DeclarePairedDelimiter{\ceil}{\lceil}{\rceil}



\begin{document}

\begin{enumerate}
\item {[10]} Величины $(X_n)$ независимы и равномерно распределены на отрезке $[1; 2]$.
\begin{enumerate}
\item {[5]} Найдите предел по вероятности
\[
\plim \frac{X_1^2 + X_2^2 + \dots + X_n^2}{n}.
\]
% \item Сходится ли последовательность дробей в пункте (а) по распределению и в пространстве $L^2$?
\item {[5]} Найдите предел по вероятности 
\[
    \plim \frac{(X_1 - X_2)^2 + (X_2 - X_3)^2 + \dots + (X_{n-1} - X_n)^2}{3n + 2025}.
\]
\end{enumerate}



1a
\begin{itemize}
    \item арифметика -1б
    \item нет упоминания ЗБЧ/упоминания независимости -2б
\end{itemize}
1б 
\begin{itemize}
    \item $\E(X_i X_{i-1}) = \E(X_i^2) -1$;
    \item верные только ответ (решение существенно не доведено) = 1б
    \item нет проверки на независимость, но расписано в виде суммы -2б [$\sum (X_i - X_{i+1})^2 = \sum X_i^2 + 2 \sum X_i X_j$ и далее работа с ней]
    \item арифметика -1б
\end{itemize}


\item {[10]} Рассмотрим две последовательности нормально распределённых случайных величин, 
\[
X_n \sim \cN((2n+1)/n; (4n^2 + 1) / n^2) \quad \text{и} \quad Y_n \sim \cN((2n + 1)/n; (4n + 1) / n^2).
\]
\begin{enumerate}
    \item {[2 + 2 + 2]} К чему сходятся по распределению последовательности $(X_n)$, $(Y_n)$ и $(X_n Y_n)$?
    \item {[2 + 2]} Если возможно, приведите пример, когда последовательность $(X_n)$ сходится по вероятности и когда она не сходится по вероятности.
\end{enumerate}

2а
\begin{itemize}
    \item арифметика -1б (за подпункт) [$2 \cN(2,4)$ считается арифметикой, нужно внести 2 внутрь);
    \item только ответ = 1б (за подпункт)
    \item $\lim Y_n \sim \cN(2,0)$ = 1б (неуказанно, что это константа)
\end{itemize}
2б
\begin{itemize}
    \item примеры +1б (если 1 пример, то 0б)
    \item док-ва +1б (если 1 док-во, то 0б)
\end{itemize}


\item {[10]} Величины $X_1$, $X_2$, $X_3$ независимы и равномерно распределены на отрезке $[1;2]$.
Найдите характеристическую функцию случайной величины $Y$,
\[
Y = \begin{cases}
    X_1, \text{ если } X_1 > 1.5 \text{ и } X_2 > 1.5, \\
    X_1 + X_2 + X_3, \text{ иначе.}
\end{cases}
\]



\begin{itemize}
    \item по 1 баллу за отдельно верно найденные харфункции $X_1$, $X_1+X_2+X_3$, вероятности событий;
    \item 5 за неверное решение вида $1/4 \phi(t)+3/4 \phi^3(t)$;
    \item 6 за решение содержащее верные слагаемые, но с потерянными случаями или косячными случаями;
    \item 9 за верное с какими-то минимальными ошибками типа неверных вероятностей;
    \item 10 - полностью верное решение
\end{itemize}

\item {[10]} Характеристическая функция величины $X$ равна $\phi(t) = \exp(2\exp(-2it))/\exp(2)$.
\begin{enumerate}
    \item {[6]} Какое распределение имеет величина $X$?
    \item {[4]} Найдите $\E(X)$ и $\Var(X)$.
\end{enumerate}
% neg poisson с маскировкой


задача 4а
\begin{itemize}
    \item 1  - обрубание тейлоровского разложения и ответ N(-4,8), ссылка на пуассоновское без указания параметра, экспоненциальное итп;
    \item 3 - Poiss(2)
    \item 4 - есть попытка преобразовать, но неверный ответ (как правило, деление на -2 вместо умножения)
    \item 6 - верный ответ
\end{itemize}

задача 4б 
\begin{itemize}
    \item 1 - табличный ответ по неверному предположению из 4а
    \item 4 - верный ответ из табличного верного в 4а или прямым вычислением
    \item штраф по -1 за арифметику при вычислении отдельно каждого пункта или за неверную формулу связи хар функции и момента
\end{itemize}



\item {[10]} Немного сигма-алгебр для настоящего самурая!
\begin{enumerate}
    \item {[2]} Множество всех исходов равно $\Omega = \{a, b, c\}$. 
    Случайная величина $Y$ определена как $Y(a) = -1$, $Y(b) = 1$, $Y(c) = 2$.
    Найдите сигма-алгебру $\sigma(\cos Y)$.
    \item {[4]} Верно ли, что $\sigma(X) \subseteq \sigma(X^2)$ для произвольной случайной величины $X$? Докажите или приведите контр-пример.
    \item {[4]} Верно ли, что $\sigma(X^2) \subseteq \sigma(X)$ для произвольной случайной величины $X$? Докажите или приведите контр-пример.
\end{enumerate}

Примечание: здесь $\sigma(R)$ — минимальная сигма-алгебра, порождённая величиной $R$, а не стандартное отклонение :)



\section*{Задача 5a}

Обозначим $X = \cos(Y)$, тогда $X(a) = X(b) = \cos(1) = \cos(-1)$ так как косинус симметричен и $X(c) = \cos(2)$. $\sigma(X)$ в данном случае будет порождена событиями $\{ w \in \Omega \mid X(w) = cos(1)\}$ и $\{ w \in \Omega \mid X(w) = cos(2)\}$, то есть $\{a, b\}$ и $\{c\}$. В любой сигма-алгебре также лежат $\emptyset$ и $\Omega$, отсюда ответ:
\[
\sigma(X) = \{ \emptyset, \Omega, \{a, b\}, \{c\}\}.
\]
Прямая проверка аксиом показывает, что это корректная сигма алгебра.

\vspace{0.3cm}

\textbf{Критерии:}

\begin{itemize}
    \item Приведенный ответ не является системой подмножеств $\Omega$ — 0 баллов за задачу.
\end{itemize}

\section*{Задача 5б}

Нет, не верно. Рассмотрим пример, похожий на пример из предыдущего пункта: $\Omega = \{a, b, c\}$ и $X(a) = 1, X(b) = -1, X(c) = 0$. Так как значения величины $X$ на всех элементарных исходах различны, $\sigma(X)$ содержит все подмножетсва $\Omega$:

\[
\sigma(X) = \{ \emptyset, \Omega, \{a\}, \{b\}, \{c\}, \{a, b\}, \{a, c\}, \{b, c\}\}.
\]
В случае $X^2$ имеем $X^2(a) = X^2(b) = 1$ и $X^2(c) = 0$, отсюда, аналогично предыдущему пункту, имеем
\[
\sigma(X^2) = \{ \emptyset, \Omega, \{a, b\}, \{c\}\}.
\]
Тогда $\sigma(X) \nsubseteq \sigma(X^2)$.

\vspace{0.3cm}

\newpage
\textbf{Критерии:}

\begin{itemize}
    \item Есть корректные рассуждения о том, почему это неверно, но отсутствует сам контр-пример случайной величины — 1 или 2 балла из 4 за задачу в зависимости от полноты рассуждений
    \item Есть корректный контрпример (подразумевает хотя бы множество элементарных исходов $\Omega$ и отображение $X$ из $\Omega$ в числа) с недочетами, напрример для него неверно выписаны $\sigma(X)$ или $\sigma(X^2)$, или нет корректного доказательства отсутствия включения — 2 или 3 балла из 4 за задачу в зависимости от масштаба неточностей
\end{itemize}


\section*{Задача 5в}

Да, верно. $\sigma(X)$ по определению является минимальной сигма-алгеброй, содержащей события вида $\{ w \in \Omega \mid X(w) \le v\}$ для всех $v \in \RR$. Эквивалентным определением является минимальная сигма-алгебра, содержащей события вида $\{ w \in \Omega \mid X(w) \in B\}$ для всех множеств $B$ из Борелевской сигма алгебры.

$\sigma(X^2)$ тогда является минимальной сигма-алгеброй, содержащей события вида $\{ w \in \Omega \mid X^2(w) \le v\}$ для всех $v \in \RR$. При $v < 0$ это будет пустое множество $\emptyset$. При $v \ge 0$ имеем $$\{ w \in \Omega \mid X^2(w) \le v\} = \{ w \in \Omega \mid -\sqrt{v} \le X(w) \le \sqrt{v}\}.$$

Отрезок $[-\sqrt{v}, \sqrt{v}]$ лежит в Борелевской сигма алгебре, а значит $\{w \in \Omega \mid -\sqrt{v} \le X(w) \le \sqrt{v}\}$ лежит в $\sigma(X)$, тогда и $\{ w \in \Omega \mid X^2(w) \le v\}$ лежит в $\sigma(X)$.

Отсюда $\sigma(X^2) \subseteq \sigma(X)$.

\vspace{0.3cm}


Частичные баллы ставились за разумные рассуждения, не являющиеся формальным доказательством, или за доказательства с недочетами.




\item {[10]} Каждый день в заезде участвую только две лошади: Юлиус и Фру-фру. 
Ставки на Фру-фру принимаются с коэффициентом $2$, то есть при победе Фру-фру ставка будет возвращена в двойном размере. 
Ставки на Юлиуса принимаются с коэффициентом $4$.
Вероятность победы Фру-фру равна $2/3$.

Игрок начинает со стартовой суммой $S_0 = 100$ и каждый день ставит все свои деньги в некоторой пропорции на Фру-фру и Юлиуса. 

Определим долгосрочную процентную ставку $r$ условием $\plim (S_n / S_0)^{1/n} = 1 + r$, где $S_n$ — благосостояние игрока после $n$ дней.
\begin{enumerate}
    \item {[2]} Какая стратегия максимизирует $\E(S_n)$?
    \item {[5]} Какая стратегия максимизирует долгосрочную процетную ставку?
    \item {[3]} Какая стратегия гарантирует безрисковый доход с $\Var(S_n) = 0$?
\end{enumerate}


Пусть каждый день мы ставим долю $v$ от имеющихся сбережений на Фру-фру и долю $(1 - v)$ на Юлиуса, $0 \le v \le 1$ (по условию мы должны каждый день поставить все свои деньги, поэтому доли обязаны суммироваться в единицу). Пускай в определенный день у нас на руках было $x$ денег. Тогда при победе Фру-фру на руках в конце дня мы будем иметь $2vx$, а при победе Юлиуса мы будем иметь $4(1-v)x$.

\vspace{0.3cm}

Введем последовательность независимых одинаково распределенных величин $M_i$, где $M_i$ принимает значение $2v$ с вероятностью $2/3$ и значение $4(1-v)$ с вероятностью $1/3$. По сути каждый день наши сбережения домножаются на случайный коэффициент. Тогда имеем 
\[
S_n = S_0 \prod_{i=1}^n M_i.
\]
В пункте а) имеем 
\[
\E[S_n] = S_0 \E\left[  \prod_{i=1}^n M_i \right] =S_0 \prod_{i=1}^n \E[M_i].
\]
В последнем равенстве мы воспользовались независимостью величин $M_i$. Отсюда для максимизации матожидания мы должны максимизировать $$\E[M_i] = \frac{2}{3}2v + \frac{1}{3}4(1 - v) = \frac{4}{3}v + \frac{4}{3}(1-v) = \frac{4}{3}.$$
Матожидание это константа, тогда нам подойдет любое $0 \le v \le 1$.

\vspace{0.3cm}

В пункте б) хотим максимизировать $\plim(S_n / S_0)^{1/n}$. По лемме о наследовании сходимости это то же самое, что максимизировать $\plim\log \left((S_n / S_0)^{1/n}\right)$, так как логарифм является монотонно возрастающей функцией. Далее
\[
\log \left((S_n / S_0)^{1/n}\right) = \frac{\log S_0}{n} + \frac{1}{n}\left( \sum_{i=1}^n \log M_i\right).
\]
Первое слагаемое это константа, а второе слагаемое по ЗБЧ сходится к $\E\left[ \log M_i\right]$ по вероятности, отсюда нам нужно максимизировать данное матожидание по $v$ (\textit{здесь мы по сути вывели критерий Кэлли, принимается просто сослаться на лекцию}). Распишем
\[
\E\left[\log M_i\right] = \frac{2}{3}\log(2v) + \frac{1}{3}\log(4(1 - v)).
\]
Найдем производную по $v$:
\[
\frac{4}{6v} - \frac{4}{12(1-v)} = \frac{2}{3v} - \frac{1}{3(1-v)}
\]
Прирванивая к нулю, получим
\[
\frac{2}{3v} = \frac{1}{3(1-v)}
\]
\[
6(1-v) = 3v
\]
\[
v = \frac{2}{3}
\]
Отсюда оптимальной стратегией будет две трети сбережений ставить на Фру-фру и одну треть на Юлиуса.

\vspace{0.3cm}

В пункте в) мы хотим добиться $\Var(S_n) = 0$. Это возможно только если $S_n$ это константа, а это возможно только если все коэффициенты $M_i$ — константы. Вспомним, что $M_i$ принимает значение $2v$ с вероятностью $2/3$ и значение $4(1-v)$ с вероятностью $1/3$. Значит мы хотим добиться $2v = 4(1 - v)$, отсюда единственное подходящее $v$ это $v = 2/3$. Значит стратегией с безрисковым доходом также будет две трети сбережений ставить на Фру-фру и одну треть на Юлиуса.

\vspace{0.3cm}

\textbf{Критерии:}

\begin{itemize}
    \item Верные рассуждения но неверно выписанное матожидание или сам коэффициент в пункте a) — 1 балл из 2 за пункт а)
    \item В пункте б) объяснено, что надо максимизировать матожидание логарифма коэффициента $\E[\log M_i]$ (вывести или сослаться на критерий Кэлли) — +2 балла за пункт б)
    \item В пункте б) корректно найдено $\E[\log M_i]$ и найден максимум по $v$ — +3 балла за пункт б) (возможны частичные баллы при неверно выписанном матожидании/неверно выписанном коэффициенте/арифметических ошибках)
    \item В пункте в) приведены корректные рассуждения достаточной степени подробности, как добиться безрискового дохода — +1 балл за пункт в) 
    \item В пункте в) найден правильный ответ — +2 балла за пункт в) 
\end{itemize}




\end{enumerate}
    


\end{document}

