% arara: xelatex
\documentclass[12pt]{article}

% \usepackage{physics}

\usepackage{hyperref}
\hypersetup{
    colorlinks=true,
    linkcolor=blue,
    filecolor=magenta,      
    urlcolor=cyan,
    pdftitle={Overleaf Example},
    pdfpagemode=FullScreen,
    }

\usepackage{tikzducks}

\usepackage{tikz} % картинки в tikz
\usetikzlibrary{shapes, arrows, positioning}
\usepackage{microtype} % свешивание пунктуации

\usepackage{array} % для столбцов фиксированной ширины

\usepackage{indentfirst} % отступ в первом параграфе

\usepackage{sectsty} % для центрирования названий частей
\allsectionsfont{\centering}

\usepackage{amsmath, amsfonts, amssymb} % куча стандартных математических плюшек

\usepackage{comment}

\usepackage[top=2cm, left=1.2cm, right=1.2cm, bottom=2cm]{geometry} % размер текста на странице

\usepackage{lastpage} % чтобы узнать номер последней страницы

\usepackage{enumitem} % дополнительные плюшки для списков
%  например \begin{enumerate}[resume] позволяет продолжить нумерацию в новом списке
\usepackage{caption}

\usepackage{url} % to use \url{link to web}


\newcommand{\smallduck}{\begin{tikzpicture}[scale=0.3]
    \duck[
        cape=black,
        hat=black,
        mask=black
    ]
    \end{tikzpicture}}

\usepackage{fancyhdr} % весёлые колонтитулы
\pagestyle{fancy}
\lhead{Теория вероятностей для самураев}
\chead{}
\rhead{Контрольная работа}
\lfoot{}
\cfoot{}
\rfoot{}


\renewcommand{\headrulewidth}{0.4pt}
\renewcommand{\footrulewidth}{0.4pt}

\usepackage{tcolorbox} % рамочки!

\usepackage{todonotes} % для вставки в документ заметок о том, что осталось сделать
% \todo{Здесь надо коэффициенты исправить}
% \missingfigure{Здесь будет Последний день Помпеи}
% \listoftodos - печатает все поставленные \todo'шки


% более красивые таблицы
\usepackage{booktabs}
% заповеди из докупентации:
% 1. Не используйте вертикальные линни
% 2. Не используйте двойные линии
% 3. Единицы измерения - в шапку таблицы
% 4. Не сокращайте .1 вместо 0.1
% 5. Повторяющееся значение повторяйте, а не говорите "то же"


\setcounter{MaxMatrixCols}{20}
% by crazy default pmatrix supports only 10 cols :)


\usepackage{fontspec}
\usepackage{libertine}
\usepackage{polyglossia}

\setmainlanguage{russian}
\setotherlanguages{english}

% download "Linux Libertine" fonts:
% http://www.linuxlibertine.org/index.php?id=91&L=1
% \setmainfont{Linux Libertine O} % or Helvetica, Arial, Cambria
% why do we need \newfontfamily:
% http://tex.stackexchange.com/questions/91507/
% \newfontfamily{\cyrillicfonttt}{Linux Libertine O}

\AddEnumerateCounter{\asbuk}{\russian@alph}{щ} % для списков с русскими буквами
% \setlist[enumerate, 2]{label=\asbuk*),ref=\asbuk*}

%% эконометрические сокращения
\DeclareMathOperator{\Cov}{\mathbb{C}ov}
\DeclareMathOperator{\Corr}{\mathbb{C}orr}
\DeclareMathOperator{\Var}{\mathbb{V}ar}
\DeclareMathOperator{\col}{col}
\DeclareMathOperator{\row}{row}

\DeclareMathOperator{\rank}{rank}

\let\P\relax
\DeclareMathOperator{\P}{\mathbb{P}}

\let\H\relax
\DeclareMathOperator{\H}{\mathbb{H}}


\DeclareMathOperator{\E}{\mathbb{E}}
% \DeclareMathOperator{\tr}{trace}
\DeclareMathOperator{\card}{card}

\DeclareMathOperator{\Convex}{Convex}
\DeclareMathOperator{\plim}{plim}

\newcommand{\cN}{\mathcal{N}}
\newcommand{\cF}{\mathcal{F}}


\newcommand{\SST}{\text{SST}}
\newcommand{\SSR}{\text{SS}^{\text{res}}}

\newcommand{\RR}{\mathbb{R}}
\newcommand{\NN}{\mathbb{N}}
\newcommand{\hb}{\hat{\beta}}
\newcommand{\dPois}{\mathrm{Pois}}

\usepackage{mathtools}
\DeclarePairedDelimiter{\norm}{\lVert}{\rVert}
\DeclarePairedDelimiter{\abs}{\lvert}{\rvert}
\DeclarePairedDelimiter{\scalp}{\langle}{\rangle}
\DeclarePairedDelimiter{\ceil}{\lceil}{\rceil}



\begin{document}

\begin{enumerate}
    \item {[10]} Случайная величина $X$ имеет функцию плотности $f(x) = \abs{x}$ на отрезке $[-1;1]$ и $0$ за его пределами.
    \begin{enumerate}
        \item {[3]} Найдите условную вероятность $\P(X > 0.5 \mid X > 0)$.
        \item {[4]} Найдите ковариацию $\Cov(X, X^3)$.
        \item {[3]} Найдите функцию плотности величины $Y = \ln \abs{X}$.
    \end{enumerate}

    \item {[10]} Илон и Маск независимо друг от друга подбрасывают правильную монетку.
    Илон подбрасывает $10$ раз, а Маск — $11$ раз. 
    У Илона выпадает случайное количество $X$ орлов, у Маска — $Y$ орлов. 

    \begin{enumerate}
        \item {[2]} Найдите вероятность $\P(X + Y = 7)$.
        \item {[4]} Найдите вероятность $\P(Y > X)$.
        \item {[4]} Найдите условное ожидание $\E(X \mid X + Y = 12)$.
    \end{enumerate}

    Подсказка: в быстром ответе на всю задачу остаётся один биномиальный коэффициент :)


    \vspace{10pt}
    Фамилия, имя и группа: \dotfill

    \newpage

    \item {[10]} Пара студентов играет один матч в камень-ножницы-бумага.
    Матч состоит из нескольких раундов.
    Все игроки всегда выбирают равновероятно камень, ножницы и бумагу. 
    Раунды играют до тех пор, пока не определится победитель. 

    Обозначим $T$ — число ничьих раундов, а $S$ — общее число ножниц в матче у обоих игроков. 
    \begin{enumerate}
        \item {[3]} Найдите энтропию $\H(T)$.
        \item {[7]} Найдите энтропию $\H(S)$.
    \end{enumerate}


    \item {[10]} Студенты фкн в составе 300 человек играют в камень-ножницы-бумага индивидуально до определения Самого Главного Везунчика.
    В каждой паре игроки играют один матч, состоящий из раундов камень-ножница-бумага до тех пор, пока не определится победитель.
    Проигравший раунд (и матч) игрок выбывает и далее в матчах не участвует. 
    Все игроки всегда выбирают равновероятно камень, ножницы и бумагу. 

    Обозначим $N$ — общее число раундов (не матчей!). 

    \begin{enumerate}
        \item {[2]} Найдите вероятность $\P(N = 300)$.
        \item {[4]} Найдите ожидание $\E(N)$.
        \item {[4]} Найдите дисперсию $\Var(N)$.
    \end{enumerate}

    \vspace{10pt}
    Фамилия, имя и группа: \dotfill

    \newpage



    

    \item {[10]} На сцене четыре закрытых двери. 
    За одной из дверей — дорогой автомобиль, за остальными — козы. 
    Ведущий шоу знает, что находится за каждой дверью, игрок шоу — не знает.
    Игрок хочет выиграть автомобиль. 
    Шоу идёт так:

    \begin{enumerate}
        \item[Шаг 1.] Игрок встаёт возле одной из закрытых дверей.
        
        \item[Шаг 2.] Ведущий открывает одну из дверей с козой, возле которой нет игрока.
        Остаётся закрытыми три двери, у одной из которых стоит игрок. 
        Затем ведущий предлагает игроку возможность перейти к любой другой двери.

        \item[Шаг 3.] Игрок перемещается или остаётся на месте.

        \item[Шаг 4.] Ведущий снова открывает одну из дверей с козой, возле которой нет игрока.
        Остаётся закрытыми две двери, у одной из которых стоит игрок.
        Снова ведущий предлагает игроку возможность перейти к другой закрытой двери.

        \item[Шаг 5.] Игрок перемещается или остаётся на месте.    
        \item[Шаг 6.] Игрок получает то, что находится за дверью, у которой он стоит.
    \end{enumerate}
    
    \begin{enumerate}
        \item {[7]} Как выглядит оптимальная стратегия игрока?
        \item {[3]} Чему равна вероятность получения автомобиля при оптимальной стратегии?
    \end{enumerate}
    

    \item {[10]} Пара величин $(X, Y)$ имеет функцию плотности $f(x, y) = 6xy^2$ на квадрате $[0;1]\times [0;1]$ и $0$ вне квадрата. 
    \begin{enumerate}
        \item {[3]} Найдите ожидание $\E(X/Y)$ и вероятностью $\P(X > Y)$.
        \item {[3]} Найдите функцию распределения $F_X(t)$.
        \item {[1]} Зависимы ли величины $X$ и $Y$?
        \item {[3]} Найдите ожидание $\E(W)$, где $W = F(X, Y)$ и $F$ — совместная функция распределения. 
    \end{enumerate}

    \vspace{10pt}
    Фамилия, имя и группа: \dotfill

\end{enumerate}

\begin{enumerate}
    \item 
    \item 
    \item Для геометрического распределения с вероятностью $p$ 
    энтропия равна $\H(X) = -\ln p - \ln (1 - p) (1 - p) / p$.

    Для $\P(T = k) = p (1 - p)^k$ для $k \in \{0, 1, 2, \dots \}$ с $p = 2/3$.
    
    Отсюда $H(T) = 1.5 \ln 3 - \ln 2$. 
    
    За верную формулу для $\P(T = k)$ — 1 балл. 


    \item 

    \begin{enumerate}
        \item Нужна одна ничья и 288 побед за один раунд, $\P(N = 300) = 299 \cdot (2/9) (2/3)^{288}$.
        
        За неверный ответ $\P(N = 300) = (2/3)^{300}$ поставил 1 балл.
        \item $\E(N) = 299 \cdot 1/p = 299 \cdot 3 /2$.
        \item $\Var(N) = 299 \cdot (1-p)/p^2 = 299 \cdot 3/4$.
    \end{enumerate}

\end{enumerate}



\end{document}

