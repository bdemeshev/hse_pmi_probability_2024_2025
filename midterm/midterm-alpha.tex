% arara: xelatex
\documentclass[12pt]{article}

% \usepackage{physics}

\usepackage{hyperref}
\hypersetup{
    colorlinks=true,
    linkcolor=blue,
    filecolor=magenta,      
    urlcolor=cyan,
    pdftitle={Overleaf Example},
    pdfpagemode=FullScreen,
    }

\usepackage{tikzducks}

\usepackage{tikz} % картинки в tikz
\usetikzlibrary{shapes, arrows, positioning}
\usepackage{microtype} % свешивание пунктуации

\usepackage{array} % для столбцов фиксированной ширины

\usepackage{indentfirst} % отступ в первом параграфе

\usepackage{sectsty} % для центрирования названий частей
\allsectionsfont{\centering}

\usepackage{amsmath, amsfonts, amssymb} % куча стандартных математических плюшек

\usepackage{comment}

\usepackage[top=2cm, left=1.2cm, right=1.2cm, bottom=2cm]{geometry} % размер текста на странице

\usepackage{lastpage} % чтобы узнать номер последней страницы

\usepackage{enumitem} % дополнительные плюшки для списков
%  например \begin{enumerate}[resume] позволяет продолжить нумерацию в новом списке
\usepackage{caption}

\usepackage{url} % to use \url{link to web}


\newcommand{\smallduck}{\begin{tikzpicture}[scale=0.3]
    \duck[
        cape=black,
        hat=black,
        mask=black
    ]
    \end{tikzpicture}}

\usepackage{fancyhdr} % весёлые колонтитулы
\pagestyle{fancy}
\lhead{Теория вероятностей для самураев}
\chead{}
\rhead{Контрольная работа}
\lfoot{}
\cfoot{}
\rfoot{}


\renewcommand{\headrulewidth}{0.4pt}
\renewcommand{\footrulewidth}{0.4pt}

\usepackage{tcolorbox} % рамочки!

\usepackage{todonotes} % для вставки в документ заметок о том, что осталось сделать
% \todo{Здесь надо коэффициенты исправить}
% \missingfigure{Здесь будет Последний день Помпеи}
% \listoftodos - печатает все поставленные \todo'шки


% более красивые таблицы
\usepackage{booktabs}
% заповеди из докупентации:
% 1. Не используйте вертикальные линни
% 2. Не используйте двойные линии
% 3. Единицы измерения - в шапку таблицы
% 4. Не сокращайте .1 вместо 0.1
% 5. Повторяющееся значение повторяйте, а не говорите "то же"


\setcounter{MaxMatrixCols}{20}
% by crazy default pmatrix supports only 10 cols :)


\usepackage{fontspec}
\usepackage{libertine}
\usepackage{polyglossia}

\setmainlanguage{russian}
\setotherlanguages{english}

% download "Linux Libertine" fonts:
% http://www.linuxlibertine.org/index.php?id=91&L=1
% \setmainfont{Linux Libertine O} % or Helvetica, Arial, Cambria
% why do we need \newfontfamily:
% http://tex.stackexchange.com/questions/91507/
% \newfontfamily{\cyrillicfonttt}{Linux Libertine O}

\AddEnumerateCounter{\asbuk}{\russian@alph}{щ} % для списков с русскими буквами
% \setlist[enumerate, 2]{label=\asbuk*),ref=\asbuk*}

%% эконометрические сокращения
\DeclareMathOperator{\Cov}{\mathbb{C}ov}
\DeclareMathOperator{\Corr}{\mathbb{C}orr}
\DeclareMathOperator{\Var}{\mathbb{V}ar}
\DeclareMathOperator{\col}{col}
\DeclareMathOperator{\row}{row}

\DeclareMathOperator{\rank}{rank}

\let\P\relax
\DeclareMathOperator{\P}{\mathbb{P}}

\let\H\relax
\DeclareMathOperator{\H}{\mathbb{H}}


\DeclareMathOperator{\E}{\mathbb{E}}
% \DeclareMathOperator{\tr}{trace}
\DeclareMathOperator{\card}{card}

\DeclareMathOperator{\Convex}{Convex}
\DeclareMathOperator{\plim}{plim}

\newcommand{\cN}{\mathcal{N}}
\newcommand{\cF}{\mathcal{F}}


\newcommand{\SST}{\text{SST}}
\newcommand{\SSR}{\text{SS}^{\text{res}}}

\newcommand{\RR}{\mathbb{R}}
\newcommand{\NN}{\mathbb{N}}
\newcommand{\hb}{\hat{\beta}}
\newcommand{\dPois}{\mathrm{Pois}}
\newcommand{\dBin}{\mathrm{Bin}}

\usepackage{mathtools}
\DeclarePairedDelimiter{\norm}{\lVert}{\rVert}
\DeclarePairedDelimiter{\abs}{\lvert}{\rvert}
\DeclarePairedDelimiter{\scalp}{\langle}{\rangle}
\DeclarePairedDelimiter{\ceil}{\lceil}{\rceil}



\begin{document}

\begin{enumerate}
    \item {[10]} Случайная величина $X$ имеет функцию плотности $f(x) = \abs{x}$ на отрезке $[-1;1]$ и $0$ за его пределами.
    \begin{enumerate}
        \item {[3]} Найдите условную вероятность $\P(X > 0.5 \mid X > 0)$.
        \item {[4]} Найдите ковариацию $\Cov(X, X^3)$.
        \item {[3]} Найдите функцию плотности величины $Y = \ln \abs{X}$.
    \end{enumerate}

    \item {[10]} Илон и Маск независимо друг от друга подбрасывают правильную монетку.
    Илон подбрасывает $10$ раз, а Маск — $11$ раз. 
    У Илона выпадает случайное количество $X$ орлов, у Маска — $Y$ орлов. 

    \begin{enumerate}
        \item {[2]} Найдите вероятность $\P(X + Y = 7)$.
        \item {[4]} Найдите вероятность $\P(Y > X)$.
        \item {[4]} Найдите условное ожидание $\E(X \mid X + Y = 12)$.
    \end{enumerate}

    Подсказка: в быстром ответе на всю задачу остаётся один биномиальный коэффициент :)



    \item {[10]} Пара студентов играет один матч в камень-ножницы-бумага.
    Матч состоит из нескольких раундов.
    Все игроки всегда выбирают равновероятно камень, ножницы и бумагу. 
    Раунды играют до тех пор, пока не определится победитель. 

    Обозначим $T$ — число ничьих раундов, а $S$ — общее число ножниц в матче у обоих игроков. 
    \begin{enumerate}
        \item {[3]} Найдите энтропию $\H(T)$.
        \item {[7]} Найдите энтропию $\H(S)$.
    \end{enumerate}


    \item {[10]} Студенты фкн в составе 300 человек играют в камень-ножницы-бумага индивидуально до определения Самого Главного Везунчика.
    В каждой паре игроки играют один матч, состоящий из раундов камень-ножница-бумага до тех пор, пока не определится победитель.
    Проигравший раунд (и матч) игрок выбывает и далее в матчах не участвует. 
    Все игроки всегда выбирают равновероятно камень, ножницы и бумагу. 

    Обозначим $N$ — общее число раундов (не матчей!). 

    \begin{enumerate}
        \item {[2]} Найдите вероятность $\P(N = 300)$.
        \item {[4]} Найдите ожидание $\E(N)$.
        \item {[4]} Найдите дисперсию $\Var(N)$.
    \end{enumerate}
    

    \item {[10]} На сцене четыре закрытых двери. 
    За одной из дверей — дорогой автомобиль, за остальными — козы. 
    Ведущий шоу знает, что находится за каждой дверью, игрок шоу — не знает.
    Игрок хочет выиграть автомобиль. 
    Шоу идёт так:

    \begin{enumerate}
        \item[Шаг 1.] Игрок встаёт возле одной из закрытых дверей.
        
        \item[Шаг 2.] Ведущий открывает одну из дверей с козой, возле которой нет игрока.
        Остаётся закрытыми три двери, у одной из которых стоит игрок. 
        Затем ведущий предлагает игроку возможность перейти к любой другой двери.

        \item[Шаг 3.] Игрок перемещается или остаётся на месте.

        \item[Шаг 4.] Ведущий снова открывает одну из дверей с козой, возле которой нет игрока.
        Остаётся закрытыми две двери, у одной из которых стоит игрок.
        Снова ведущий предлагает игроку возможность перейти к другой закрытой двери.

        \item[Шаг 5.] Игрок перемещается или остаётся на месте.    
        \item[Шаг 6.] Игрок получает то, что находится за дверью, у которой он стоит.
    \end{enumerate}
    
    \begin{enumerate}
        \item {[7]} Как выглядит оптимальная стратегия игрока?
        \item {[3]} Чему равна вероятность получения автомобиля при оптимальной стратегии?
    \end{enumerate}
    

    \item {[10]} Пара величин $(X, Y)$ имеет функцию плотности $f(x, y) = 6xy^2$ на квадрате $[0;1]\times [0;1]$ и $0$ вне квадрата. 
    \begin{enumerate}
        \item {[3]} Найдите ожидание $\E(X/Y)$ и вероятностью $\P(X > Y)$.
        \item {[3]} Найдите функцию распределения $F_X(t)$.
        \item {[1]} Зависимы ли величины $X$ и $Y$?
        \item {[3]} Найдите ожидание $\E(W)$, где $W = F(X, Y)$ и $F$ — совместная функция распределения. 
    \end{enumerate}

\end{enumerate}

Критерии и частичные ответы с решениями:

\begin{enumerate}
    \item 
    \begin{enumerate}
        \item 3 - верно;
        2 -  арифметическая ошибка при подсчете одного из интегралов;
        1 - есть верная формула для условной вероятности, но сами вероятности вычесляются неверно (не та плотность интегрируется и тому подобное);
        0,5 - нет верной формулы полной вероятности, но удалось верно посчитать какую-то из двух нужных вероятностей. Как правило, это отностися к тем, кто вместо условной вероятности посчитал просто $\P(X>0,5)$.
    
    \[
    \P(X>0.5 \mid X>0)=\frac{\P(X>0.5, X>0)}{\P(X>0)}=\frac{\P(X>0.5)}{\P(X>0)}.
    \]
    $\P(X>0)=1/2$ 
    в силу четности плотности, при желании можно и явно посчитать: 
    \[
    \P(X>0)=\int_0^{+\infty}f(x)dx=\int_0^{1}xdx=1/2.
    \]
\[
\P(X>0.5)=\int_{0.5}^{+\infty}f(x)dx=\int_{0.5}^{1}xdx=3/8.
\]
Таким образом, $\P(X>0.5 \mid X>0)=3/4$.

\item 4 -  верно;
штраф по (-1б) за каждое неверно посчитанное матожидание;
1  - есть верно написанная формула для ковариации.

$\Cov(X, X^3)=\E(X\cdot X^3) - \E X\cdot \E (X^3)=\E (X^4) - \E (X) \cdot \E (X^3)$.

$\E X = 0$ в силу четности плотности, действительно: $\int_{\RR} xf(x)dx=\int_{-1}^1 x \abs{x} dx$ — интеграл берется от нечетной функции по симметричному промежутку, значит он равен нулю.

\[
\E (X^4)=\int_{\RR} x^4f(x)dx=\int_{-1}^1 x^4\abs{x}dx=[\text{в силу четности и симметрии промежутка}] = 2\int_{0}^1 x^4\cdot xdx=1/3 
\]
Таким образом, $\Cov(X,X^3) = 1/3-0=1/3$.

\item 
\textbf{Решение через поиск Ф.Р.: } Для начала найдем функцию распределения $Y$:

\[
F_Y(y) = \P(Y\leq y) = \P(\ln \abs{x}\leq y) = \P(\abs{x}\leq e^y) = \P(-e^y\leq X \leq e^y)=
\begin{cases}
F_X(e^y)-F_x(-e^y),& y\leq 0\\
1,& y>0
\end{cases}
\]
Теперь найдем плотность, помня о том, что $F'_x(x)=f_x(x)$ и о формуле производной сложной функции: 
\[
f_Y(y)=F'_y(y)=
\begin{cases}
f_X(e^y)\cdot e^y-f_X(-e^y)\cdot(-e^y)=2e^{2y},& y\leq 0\\
0,& y>0
\end{cases}
\]

3 - верно;
2 - ФР Y верное выражена через ФР Х, но при подсчете плотности потерялась производная сложной функции и, соответственно, двойка в степени. либо была ошибка при выражении ФР, в результате потерялась двойка перед экспонентой в плотности, потеряно, что на положительных y плотность нулевая;
1 - есть верная связь ФР Y c ФР X, записанная через вероятность. Дальше неверно найдена ФР Х, если она искалась, либо неверно сделан сразу переход к плотностям, либо нет дальнейших продвижений;
0,5 - есть понимание связи между случайными величинами (предлагается считать плостность у как плостность х в точке $e^y$).


\textbf{Решение через o-малые: } Рассмотрим 
\begin{multline*}
    \P(Y\in [y,y+h])=P(\ln \abs{x} \in [y,y+h]) = \P(X\in[e^y,e^{y+h}] )+ \P(X\in[-e^y,-e^{y+h}] )= \\
    =[\text{в силу четности плотности}]=2\P(X\in[e^y, e^{y+h}])=
    \begin{cases} 
        2 \abs{e^y} \cdot (e^{y+h}-e^y)+o(e^{y+h}-e^y),& y\leq 0\\
        o(e^{y+h}-e^y),& y>0
    \end{cases}  
\end{multline*}

Преобразуем последнее выражение: $(e^{y+h}-e^y)=e^y(e^h-1)=e^y(h+o(h))$, таким образом, $o(e^{y+h}-e^y)=o(h)$  и имеем
\[
\P(Y\in [y,y+h])=
\begin{cases}
2 e^y \cdot (e^y h+o(h))+o(h)=2e^{2y}h+o(h),& y\leq 0\\
o(h),& y>0,
\end{cases}
\]
то есть $f_Y(y)=
\begin{cases}
2e^{2y},& y\leq 0\\
0,& y>0.
\end{cases}$.
\end{enumerate}

3 - верно;
2 - верно написана вероятность попадания в нужный отрезок, но возникла проблема с работой с о-малыми при попытке привести к нужному виду;
1 - есть попытка записать нужную вероятность через о-малые, но нет понимания, что нужно перейти от $\delta x$ к $\delta y$;


\item 
\begin{enumerate}
\item 
2 - верно;
1 - ответ оставлен в виде суммы произведений биномиальных коэффициентов (либо эта сумма посчитана неверно), либо вместо искомой вероятности посчитано количество комбинаций, либо при подсчете через производящие функции ошибка из-за потери коэффициентов;
$0.5$ - вычислены какие-то из вероятностей вида $\P(x=k,y=7-k)$;
0 - рассуждение, в котором считается, что все комбинации вида (x=k,y=7-k) равновероятны и предлагается считать вероятность как их количество, деленное на общее число возможных пар значений (то есть подсчеты типа $8/(11 \cdot 12)$;

Заметим, что величина $X+Y$ соответствует количеству орлов, выпавших при 21 независимом броске монетки, 
то есть имеет биномиальное распределение $X+Y\sim \dBin(21,1/2)$. 
А значит 
\[
\P(X+Y=7)=C_{21}^7(1/2)^7(1/2)^{14}=C_{21}^7(1/2)^{21}.
\]

\item 4 - верно получен ответ 1/2;
3 - есть небольшая дыра в обосновании равновероятности событий $(Y>X)$ и $(Y \leq X)$;
2 - ответ оставлен в виде двойной суммы биномиальных коэффициентов;
1 - ответ сведен к двойной сумме вероятностей событий, но эти вероятности не вычислены, либо вместо вероятностей посчитаны количества и ответ оставлен в виде двойной суммы;
0 - рассуждение, в котором считается, что все комбинации вида (x=k,y=m) равновероятны и предлагается считать вероятность как их количество, деленное на общее число возможных пар значений;

Покажем, что $\P(Y>X) = \P(Y\leq X)$, а значит равна $1/2$.
Для этого построим биекцию между исходами из множества $(Y>X)$ и $(Y\leq X)$, перевернув все монетки. 
Действительно, пусть в исходной полученной последовательности было $X=x$, $Y=y$ орлов, $y>x$. 
Тогда после переворачивания монеток орлы с решками поменяются местами и получится последовательность, 
в которой $X=10-x$, $Y=11-y$, $x<y$, значит $-x>-y$, $10-x>10-y$, а значит $10-x\geq 11-y$, то есть $X\geq Y$. 
Биекция между последовательностями построена, 
все возможные последовательности орлов и решек равновероятны (каждая последовательность имеет вероятность $(1/2)^{21}$), значит $\P(Y > X) = \P(Y\leq X)=1/2$.

\item  4 - верно;
3 - имеется дыра в обосновании/отсутствует обоснование условных матожиданий для отдельных монеток;
2 - Есть верная формула для подсчета УМО через суммы произведений биномиальных коэффициентов, но до числового ответа не доведено;
$1.5$ - формула из «2» с небольшими недостатками;
1 - выписана верная формула для подсчета УМО;

Поставим $j$-й монетке из 21 случайную величину $I_j$, равной 1, если на этой монете выпал орел, и 0, если выпала решка. 
Заметим, что $X+Y=I_1+\cdots+I_{21}$, $X=I_1+\cdots +I_{10}$. 
Воспользуемся линейностью ожидания: $\E(X+Y \mid X+Y=12)=12$, так как $X+Y$ константа, но  
\[
12=\E(X+Y \mid X+Y=12)=\E(I_1+\cdots+I_{21} \mid X+Y=12) = 21\E(I_1 \mid X+Y=12),
\]
то есть $\E(I_1 \mid X+Y=12)=12/21$. Но тогда 
\[
\E(X \mid X+Y=12) = \E(I_1+\cdots+I_{10} \mid X+Y=12) = 10 \E(I_1 \mid X + Y = 12) = 120/21.
\]

\end{enumerate}
\item Для геометрического распределения с вероятностью $p$ 
    энтропия равна $\H(X) = -\ln p - \ln (1 - p) (1 - p) / p$.

    Для $\P(T = k) = p (1 - p)^k$ для $k \in \{0, 1, 2, \dots \}$ с $p = 2/3$.
    
    Отсюда $H(T) = 1.5 \ln 3 - \ln 2$. 
    
    За верную формулу для $\P(T = k)$ — 1 балл. 

    Для $S$ важна чётность, $\P(S = 2k) = (4/7)(1/7)^k$,
    $\P(S = 2k + 1) = (2/7) (1/7)^k$.

    \item 

    \begin{enumerate}
        \item По правилам одно поражение приводит к выбыванию ровно одного игрока.
        Как конкретно происходит сопоставление игроков — совершенно неважно. 
        Матчей всегда будет сыграно 299 при любой схеме сопоставления. 

        Нужна одна ничья с последующей победой и 288 побед за один раунд, $\P(N = 300) = 299 \cdot (2/9) (2/3)^{288}$.
        
        За неверный ответ $\P(N = 300) = (2/3)^{300}$ ставил 1 балл.
        \item $\E(N) = 299 \cdot 1/p = 299 \cdot 3 /2$.
        \item $\Var(N) = 299 \cdot (1-p)/p^2 = 299 \cdot 3/4$.
    \end{enumerate}

    \item Оптимальная стратегия: стоять, затем менять. Даёт вероятность выигрыша $3/4$. 
    \begin{enumerate}
    \item 1 за чёткую тривиальную или нетривиальную(без объяснений) стратегию или 2 за нетривиальную с объяснением + до 5 за доп. обоснования.
    \item 1 за какие-то правильные вероятности, 2 за правильную вероятность нетривиальной неоптимальной стратегии, 3 за правильную оптимальную. 
    \end{enumerate}
    
    \item 
    \begin{enumerate}
    \item  1.5 за каждый результат -0.2 за арифметику.
    \item  3 за правильную функцию распределения, с учётом 1 и 0.
    2 без учёта 1 и 0.
    1 за учёт 1 и 0, но неправильно в середине.
    0 если плотность вместо функции распределения или аномальные интегралы.
    \item 1 если правильный критерий независимости.
    0.2 если показана некоррелированность.
    0 если получилась корреляция или странные объяснения или только ответ (ака "Да.").
    \item 3 за правильный результат и вычисление F.
    1 только за вычисление F.
    -0.2 за арифметику.
    КРОМЕ ТОГО: во всей задаче кроме пункта (b), где есть отдельный критерий, не более одного раза снимается 0.2 за не указание граничных вещей (типа, что плотность или ф. распределения не везде ненулевая).
    \end{enumerate}

\end{enumerate}



\end{document}

