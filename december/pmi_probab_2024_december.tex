% arara: xelatex
\documentclass[12pt]{article}

% \usepackage{physics}

\usepackage{hyperref}
\hypersetup{
    colorlinks=true,
    linkcolor=blue,
    filecolor=magenta,      
    urlcolor=cyan,
    pdftitle={Overleaf Example},
    pdfpagemode=FullScreen,
    }

\usepackage{tikzducks}

\usepackage{tikz} % картинки в tikz
\usetikzlibrary{shapes, arrows, positioning}
\usepackage{microtype} % свешивание пунктуации

\usepackage{array} % для столбцов фиксированной ширины

\usepackage{indentfirst} % отступ в первом параграфе

\usepackage{sectsty} % для центрирования названий частей
\allsectionsfont{\centering}

\usepackage{amsmath, amsfonts, amssymb} % куча стандартных математических плюшек

\usepackage{comment}

\usepackage[top=2cm, left=1.2cm, right=1.2cm, bottom=2cm]{geometry} % размер текста на странице

\usepackage{lastpage} % чтобы узнать номер последней страницы

\usepackage{enumitem} % дополнительные плюшки для списков
%  например \begin{enumerate}[resume] позволяет продолжить нумерацию в новом списке
\usepackage{caption}

\usepackage{url} % to use \url{link to web}


\newcommand{\smallduck}{\begin{tikzpicture}[scale=0.3]
    \duck[
        cape=black,
        hat=black,
        mask=black
    ]
    \end{tikzpicture}}

\usepackage{fancyhdr} % весёлые колонтитулы
\pagestyle{fancy}
\lhead{Теория вероятностей для самураев}
\chead{}
\rhead{Экзамен}
\lfoot{$F(0.6) = 0.73$, $F(0.4) = 0.66$, $F(-0.25) = 0.4$, $F(1.03) = 0.85$, $F(0.05) = 0.52$ }
\cfoot{}
\rfoot{}




\renewcommand{\headrulewidth}{0.4pt}
\renewcommand{\footrulewidth}{0.4pt}

\usepackage{tcolorbox} % рамочки!

\usepackage{todonotes} % для вставки в документ заметок о том, что осталось сделать
% \todo{Здесь надо коэффициенты исправить}
% \missingfigure{Здесь будет Последний день Помпеи}
% \listoftodos - печатает все поставленные \todo'шки


% более красивые таблицы
\usepackage{booktabs}
% заповеди из докупентации:
% 1. Не используйте вертикальные линни
% 2. Не используйте двойные линии
% 3. Единицы измерения - в шапку таблицы
% 4. Не сокращайте .1 вместо 0.1
% 5. Повторяющееся значение повторяйте, а не говорите "то же"


\setcounter{MaxMatrixCols}{20}
% by crazy default pmatrix supports only 10 cols :)


\usepackage{fontspec}
\usepackage{libertine}
\usepackage{polyglossia}

\setmainlanguage{russian}
\setotherlanguages{english}

% download "Linux Libertine" fonts:
% http://www.linuxlibertine.org/index.php?id=91&L=1
% \setmainfont{Linux Libertine O} % or Helvetica, Arial, Cambria
% why do we need \newfontfamily:
% http://tex.stackexchange.com/questions/91507/
% \newfontfamily{\cyrillicfonttt}{Linux Libertine O}

\AddEnumerateCounter{\asbuk}{\russian@alph}{щ} % для списков с русскими буквами
% \setlist[enumerate, 2]{label=\asbuk*),ref=\asbuk*}

%% эконометрические сокращения
\DeclareMathOperator{\Cov}{\mathbb{C}ov}
\DeclareMathOperator{\Corr}{\mathbb{C}orr}
\DeclareMathOperator{\Var}{\mathbb{V}ar}
\DeclareMathOperator{\col}{col}
\DeclareMathOperator{\row}{row}

\DeclareMathOperator{\rank}{rank}

\let\P\relax
\DeclareMathOperator{\P}{\mathbb{P}}

\let\H\relax
\DeclareMathOperator{\H}{\mathbb{H}}


\DeclareMathOperator{\E}{\mathbb{E}}
% \DeclareMathOperator{\tr}{trace}
\DeclareMathOperator{\card}{card}

\DeclareMathOperator{\Convex}{Convex}
\DeclareMathOperator{\plim}{plim}

\newcommand{\cN}{\mathcal{N}}
\newcommand{\cF}{\mathcal{F}}


\newcommand{\SST}{\text{SST}}
\newcommand{\SSR}{\text{SS}^{\text{res}}}

\newcommand{\RR}{\mathbb{R}}
\newcommand{\NN}{\mathbb{N}}
\newcommand{\hb}{\hat{\beta}}
\newcommand{\dPois}{\mathrm{Pois}}
\newcommand{\dBin}{\mathrm{Bin}}

\usepackage{mathtools}
\DeclarePairedDelimiter{\norm}{\lVert}{\rVert}
\DeclarePairedDelimiter{\abs}{\lvert}{\rvert}
\DeclarePairedDelimiter{\scalp}{\langle}{\rangle}
\DeclarePairedDelimiter{\ceil}{\lceil}{\rceil}



\begin{document}

\begin{enumerate}
    \item {[10]} Совместные расходы Императора и Императрицы $(X, Y)$ равномерно распределены в области, заданной неравенствами $0 \leq Y \leq 1$, $0 \leq X \leq 2$, $X + Y \leq 2$.
    
    \begin{enumerate}
        \item {[3]} Найдите функцию плотности расходов Императора $X$.
        \item {[3 + 4]} Найдите условные ожидание $\E(X \mid Y)$ и дисперсию $\Var(X \mid Y)$.
    \end{enumerate}

    \item {[10]} Сила удара меча у опытного самурая равномерно распределена на отрезке $[2; 5]$,
    а у неопытного — равномерно на отрезке $[1; 4]$.
    Собрались как-то вместе 500 самураев и ударили мечом по разу, независимо друг от друга. 

    \begin{enumerate}
        \item {[4]} Какова вероятность того, что суммарная сила всех ударов превысит $1500$, если среди самураев $200$ опытных? 
        \item {[6]} Сколько было опытных самураев, если вероятность того, что суммарная сила ударов опытных превзойдёт суммарную силу неопытных равна $0.6$? 
    \end{enumerate}
    
    \item {[10]} Известно, что $X \sim \cN(1, 2)$, $(Y \mid X) \sim \cN(2X, 4)$.
    \begin{enumerate}
        \item {[7]} Найдите условное распределение $(X \mid Y)$.
        \item {[3]} Найдите ковариацию $\Cov(X^2, Y)$.
    \end{enumerate}
    
    \item {[10]} Телефонные звонки поступают сёгуну Минамото-но Ёритомо согласно пуассоновскому потоку с интенсивностью $\lambda$. 
    Сёгун сегодня немного рассеян и берёт трубку на каждый звонок независимо от других с вероятностью $p$.
    Рассмотрим процесс $(Y_t)$ количества телефонных звонков, отвеченных Ёримото к моменту времени $t$.
    \begin{enumerate}
        \item {[7]} Используя аксиомы пуассоновского потока, докажите, что $(Y_t)$ — пуассоновский поток.
        \item {[3]} Какова вероятность того, что между первым поступившим звонком и первым принятым звонком пройдёт не менее получаса при $\lambda = 4$ звонка в час и $p = 0.5$?
    \end{enumerate}
    
    \item {[10]} Десять самураев случайно независимо друг от друга встали равномерно вдоль аллеи единичной длины из криптомерий. 
    Обозначим $S_1$, $S_2$, \dots, $S_{10}$ их координаты в порядке возрастания и определим вектор $W = (S_3, S_5, S_9)$.
    \begin{enumerate}
        \item {[4]} Найдите совместную функцию плотности вектора $W$.
        \item {[2 + 4]} Найдите вектор ожиданий $\E(W)$ и ковариационную матрицу $\Var(W)$.
    \end{enumerate}

    \item {[10]} Сёгун Минамото-но Ёритомо хочет отбирать на службу только опытных самураев. 
    Сила удара меча у опытного самурая равномерно распределена на отрезке $[2; 5]$,
    а у неопытного — равномерно на отрезке $[1; 4]$.

    Испытуемый самурай бьёт мечом и если сила удара оказалась больше порога $a$, то Ёримото принимает самурая на работу.

    Существует два типа ошибок. 
    Ошибка первого рода: на работу взяли неопытного самурая. 
    Ошибка второго рода: опытному самураю отказали в работе. 

    \begin{enumerate}
        \item {[3]} Найдите вероятности ошибок первого и второго рода при $a = 3.5$.
        \item {[7]} Постройте кривую зависимости ошибки второго рода от ошибки первого рода при различных порогах $a$.
    \end{enumerate}


\end{enumerate}
    

\end{document}

