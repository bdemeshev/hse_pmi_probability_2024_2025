% arara: xelatex
\documentclass[12pt]{article}

% \usepackage{physics}

\usepackage{hyperref}
\hypersetup{
    colorlinks=true,
    linkcolor=blue,
    filecolor=magenta,      
    urlcolor=cyan,
    pdftitle={Overleaf Example},
    pdfpagemode=FullScreen,
    }
\urlstyle{same}

\usepackage{tikzducks}

\usepackage{tikz} % картинки в tikz
\usepackage{microtype} % свешивание пунктуации

\usepackage{array} % для столбцов фиксированной ширины

\usepackage{indentfirst} % отступ в первом параграфе

\usepackage{sectsty} % для центрирования названий частей
\allsectionsfont{\centering}

\usepackage{amsmath, amsfonts, amssymb} % куча стандартных математических плюшек

\usepackage{mathtools}
\usepackage{comment}

\usepackage[top=2cm, left=1.2cm, right=1.2cm, bottom=2cm]{geometry} % размер текста на странице

\usepackage{lastpage} % чтобы узнать номер последней страницы

\usepackage{enumitem} % дополнительные плюшки для списков
%  например \begin{enumerate}[resume] позволяет продолжить нумерацию в новом списке
\usepackage{caption}

\usepackage{url} % to use \url{link to web}


\newcommand{\smallduck}{\begin{tikzpicture}[scale=0.3]
    \duck[
        cape=black,
        hat=black,
        mask=black
    ]
    \end{tikzpicture}}

\usepackage{fancyhdr} % весёлые колонтитулы
\pagestyle{fancy}
\lhead{}
\chead{}
\rhead{Демо версии на экзамен 2025}
\lfoot{}
\cfoot{}
\rfoot{}

\renewcommand{\headrulewidth}{0.4pt}
\renewcommand{\footrulewidth}{0.4pt}

\usepackage{tcolorbox} % рамочки!

\usepackage{todonotes} % для вставки в документ заметок о том, что осталось сделать
% \todo{Здесь надо коэффициенты исправить}
% \missingfigure{Здесь будет Последний день Помпеи}
% \listoftodos - печатает все поставленные \todo'шки


% более красивые таблицы
\usepackage{booktabs}
% заповеди из докупентации:
% 1. Не используйте вертикальные линни
% 2. Не используйте двойные линии
% 3. Единицы измерения - в шапку таблицы
% 4. Не сокращайте .1 вместо 0.1
% 5. Повторяющееся значение повторяйте, а не говорите "то же"


\setcounter{MaxMatrixCols}{20}
% by crazy default pmatrix supports only 10 cols :)


\usepackage{fontspec}
\usepackage{libertine}
\usepackage{polyglossia}

\setmainlanguage{russian}
\setotherlanguages{english}

% download "Linux Libertine" fonts:
% http://www.linuxlibertine.org/index.php?id=91&L=1
% \setmainfont{Linux Libertine O} % or Helvetica, Arial, Cambria
% why do we need \newfontfamily:
% http://tex.stackexchange.com/questions/91507/
% \newfontfamily{\cyrillicfonttt}{Linux Libertine O}

\AddEnumerateCounter{\asbuk}{\russian@alph}{щ} % для списков с русскими буквами
\setlist[enumerate, 2]{label=\asbuk*),ref=\asbuk*}

%% эконометрические сокращения
\DeclareMathOperator{\Cov}{\mathbb{C}ov}
\DeclareMathOperator{\Corr}{\mathbb{C}orr}
\DeclareMathOperator{\Var}{\mathbb{V}ar}
\DeclareMathOperator{\pCorr}{\mathrm{pCorr}}
\DeclareMathOperator{\col}{col}
\DeclareMathOperator{\row}{row}

\let\P\relax
\DeclareMathOperator{\P}{\mathbb{P}}

\DeclarePairedDelimiter{\abs}{\lvert}{\rvert}
\DeclarePairedDelimiter{\scalp}{\langle}{\rangle}

\let\H\relax
\DeclareMathOperator{\H}{\mathbb{H}}
\DeclareMathOperator{\plim}{plim}

\DeclareMathOperator{\E}{\mathbb{E}}
% \DeclareMathOperator{\tr}{trace}
\DeclareMathOperator{\card}{card}

\DeclareMathOperator{\Convex}{Convex}

\newcommand \cN{\mathcal{N}}
\newcommand \dN{\mathcal{N}}


\newcommand \RR{\mathbb{R}}
\newcommand \NN{\mathbb{N}}

\newcommand{\dBern}{\mathrm{Bern}}
\newcommand{\dBin}{\mathrm{Bin}}
\newcommand{\dGamma}{\mathrm{Gamma}}
\newcommand{\dBeta}{\mathrm{Beta}}
\newcommand{\dPois}{\mathrm{Pois}}



\begin{document}

\section*{Формат}

В экзамене будет 6 задач: четыре задачи по темам второго семестра и две — по темам первого. 
В демо версиях сделан акцент на темы второго семестра.
Задачи имеют равный вес. 
Продолжительность работы 120 минут. 
Можно будет использовать в качестве разрешенной шпаргалки один лист А4 со всех шести его сторон.


\section*{Вариант «Лискевич»}
\begin{enumerate}
    \item 
    Рассмотрим стандартный винеровский процесс $(W_t)$.
    \begin{enumerate}
        \item Найдите $\E(W_4 \mid W_5)$, $\E(W_5 \mid W_4)$, $\Var(W_4 \mid W_5)$, $\Var(W_5 \mid W_4)$.
        \item При каком $\alpha$ процесс $\exp(6W_t + \alpha t)$ будет мартингалом?
    \end{enumerate}
    
    
    \item Процессы $(W_t)$ и $(V_t)$ — стандартные винеровский процессы, независимые между собой. 
    Если возможно, найдите все такие $\alpha$ и $\beta$, чтобы процессы $(X_t)$ и $(Y_t)$ были
    стандартными винеровскими
    \[
    X_t = \alpha W_t + (1 - \alpha) V_t,  \quad Y_t = \cos(42) W_t + \sin (\beta) V_t.
    \]

    \item На первом шаге мы случайно выбираем $X$ по равномерному закону на отрезке $[0;2]$.
    На втором шаге мы случайно выбираем $Y$ по Пуассону с интенсивностью $\lambda = X$.
    \begin{enumerate}
        \item Найдите $\E(Y)$ и $\Var(Y)$.
        \item Найдите функцию плотности случайной величины $\Var(Y \mid X)$.
    \end{enumerate}

    \item Илон Маск каждый день зарабатывает случайное количество DOGE-койнов $Y_t$, экспоненициально распределённое с интенсивностью $1/10^6$.
    Заработки за разные дни независимы.
    
    Обозначим за $\tau$ тот день, когда его заработок впервые превысит $10^6$ DOGE,
    а суммарный заработок — за  $S = Y_1 + Y_2 + \dots + Y_\tau$.
    \begin{enumerate}
        \item Как распределена величина $\tau$? Найдите $\E(\tau)$.
        \item Найдите $\alpha$, чтобы процесс $M_t = \sum_{k=1}^t Y_k - \alpha t$ был мартингалом.
        \item Найдите $\E(S)$.
    \end{enumerate}
    
    \item Неправильная монетка выпадает орлом с вероятностью $p = 0.3$. 
    При выпадении орла игрок зарабатывает $X_t = +1$, а при выпадении решки — $X_t = -1$.
    Обозначим суммарный выигрыш игрока как $S_t = X_1 + X_2 + \dots + X_t$ и $\tau$ — первый момент времени, когда $S_t$ достигнет $100$ или $-50$.

    \begin{enumerate}
        \item Найдите $\alpha$ такое, что процесс $M_t = S_t - \alpha t$ — мартингал. 
        \item Найдите $\beta$ такое, что процесс $Y_t = \exp(\beta S_t)$ — мартингал. 
        \item Найдите $\P(S_\tau = 100)$.
        \item Найдите $\E(\tau)$.
    \end{enumerate}

    Подсказка: достаточно применить теорему Дуба к $M_t$ и $Y_t$.

    \item В одной корзине лежат бильярдные шары с номерами от 3 до 9, во второй — с номерами от 1 до 7. 
    Мы выбираем случайно равновероятно один шар из первой корзины и один шар — из второй. 
    Из полученных двух шаров мы равновероятно один называем $X$, а второй — $Y$.
    \begin{enumerate}
        \item Найдите $\E(Y \mid X)$.
        \item Найдите $\Var(Y \mid X)$.
    \end{enumerate}
    
%    Уточнение: можно опираться на центральную предельную теорему и леммы Слуцкого. 

\end{enumerate}


\section*{Вариант «Рафаэль»}

\begin{enumerate}

\item Рассмотрим дискретное время $t$, фильтрацию $(\mathcal{F}_t)$ и некую случайную величину $\tau$, принимающую значения из множества $\{0, 1, 2, \dots\} \cup \{\infty\}$.

Какие из приведённых условий эквивалентны, какие являются следствием других?
\begin{enumerate}[label=\Alph*:]
    \item $\forall t \in \{0, 1, 2, \dots \}: \, \{\tau \leq t\} \in \mathcal{F}_t$
    \item $\forall t \in \{0, 1, 2, \dots \}: \, \{\tau = t\} \in \mathcal{F}_t$
    \item $\forall t \in \{0, 1, 2, \dots \}: \, \{\tau < t\} \in \mathcal{F}_t$
\end{enumerate}


\item Макака снова нажимает равновероятно кнопки от А до Я на печатающей машинке.
Конец света наступает, когда макака впервые напечатает слово «АБРАКАДАБРА»,
обозначим этот момент величиной $\tau$.

\begin{enumerate}
    \item Сконструируйте мартингал, позволяющий найти $\E(\tau^2)$.
    \item Найдите $\E(\tau^2)$.
\end{enumerate}

Подсказка: если в момент $t$ добавлять в казино $t^2 - (t-1)^2$ рублей, то к моменту $t$ в казино окажется $t^2$ рублей,
\url{https://www.jeremykun.com/2014/03/03/martingales-and-the-optional-stopping-theorem/}.


\item Величины $X_1$, $X_2$, \dots, $X_n$ независимы и равномерно распределены на отрезке $[0, a]$,
рассмотрим величину $Y = \max\{X_1, \dots, X_n\}$ и её ожидание $h(a) = \E(Y)$.
\begin{enumerate}
    \item Выпишите уравнение, связывающее $h(a + u)$ и $h(a)$, с точностью до $o(u)$.
    \item Выпишите дифференциальное уравнение, которому удовлетворяет функция $h(a)$.
    \item Укажите начальное условие, которому удовлетворяет функция $h(a)$.
\end{enumerate}


\item Величины $X_1$, $X_2$, \dots, $X_n$ независимы и имеют гамма-распределение $\dGamma(\alpha, \lambda)$.
Мы складываем случайное количество слагаемых $N$, где $N$ независима от $(X_i)$ и имеет пуассоновское распределение $\dPois(\mu)$.
Получаемую сумму обозначим $S = \sum_{k=1}^N X_k$.

\begin{enumerate}
    \item Найдите $\E(\exp(u S) \mid N)$.
    \item Найдите функцию, производяющую моменты величины $S$.
\end{enumerate}

Комментарий: функцию, производящую моменты гамма-распределения можно считать известной.

\item Величины $X_1$ и $X_2$ независимы и экспоненциально распределены с параметром $\lambda$. 
\begin{enumerate}
    \item Найдите закон распреления $Y_1 = \exp(-X_1)$.
    \item Найдите функцию плотности величины $X_1 - X_2$.
    \item Найдите функцию плотности величины $\abs{X_1}$.
\end{enumerate}


\item Аня, Бэлла, Вова и Дима учатся в одной группе. 
Два студента в любой паре общаются друг с другом с вероятностью $p$ независимо от других пар.
Если студенты общаются, то любой слух, известный одному, будет известен другому. 
\begin{enumerate}
    \item Какова вероятность того, что слух дойдёт до Димы, если Аня только что узнала новый слух?
    \item Какова вероятность того, что слух дойдёт до Димы, если Аня только что узнала новый слух и не общается с Бэллой?
    \item Какова вероятность того, что слух дойдёт до Димы, если Аня только что узнала новый слух и Бэлла не общается с Вовой?
\end{enumerate}

\end{enumerate}


Немножко ответов и подсказок:



Лискевич:
\begin{enumerate}
    \item 
    \item $\alpha \in \{0, 1\}$, $\cos^2(42) + \sin^2\beta = 1$.
    \item 
    \item 
    \item 
    \item 
\[
\E(Y \mid X) = \begin{cases}
    6, \text{ если } X \in \{1, 2\},\\
    5, \text{ если } X \in \{3, 4, 5, 6, 7\},\\
    4, \text{ если } X \in \{8, 9\},\\
\end{cases}
\]
\end{enumerate}


Рафаэль:

\begin{enumerate}
    \item $A$ эквивлентно $B$, из $A$ следует $C$;
    \item Первый игрок приносит в казино $1$ рубль, второй — $3$ рубля, третий — $5$ рублей, \dots
    Внутри казино игроки играют в справедливую игру. 
    Каждый игрок в каждый момент времени ставит все деньги на очередную букву слова АБРАКАДАБРА.
    Мартингалом будет $M_t = S_t - t^2$, где $S_t$ — сумма денег у игроков внутри казино,
    \[
    S_{\tau} = 33(2\tau - 1) + 33^4(2(\tau - 3) - 1) + 33^{11}(2(\tau - 10) - 1).
    \]
    По теореме Дуба $\E(\tau^2) = \E(S_\tau)$.
    \item Величина $u$ мала. Разобъём отрезок $[0, a + u]$ на две части: большую, $[0, a]$, и малую, $[a, a + u]$.
    Либо все $n$ величин лягут на отрезок $[0, a]$, либо одна величина ляжет на малую часть $[a, a + u]$.
    Вероятностью того, что две и более величин лягут на малую часть отрезка можно пренебречь:
    \[
    h(a + u) = \left(\frac{a}{a + u}\right)^n h(a) + C_n^1 \left(\frac{a}{a + u}\right)^{n-1} \left(\frac{u}{a + u}\right) \cdot a + o(u).
    \]
\end{enumerate}

\end{document}

%Из определения условного ожидания, $\E(X \mid A) = \E(X \cdot I_A) / \P(A)$, 
%легко получаются определения условной дисперсии, $\Var(X \mid A) = \E(X^2 \mid A) - (\E(X \mid A))^2$,
%условной ковариации $\Cov(X, Y \mid A) = \E(XY \mid A) - \E(X \mid A) \E(Y \mid A)$ и даже корреляции, 
%$\Corr(X, Y \mid A) = \Cov(X, Y \mid A) / \sqrt{\Var(X \mid A)\Var(Y \mid A)}$.

% здесь проектируемая часть



\end{document}

