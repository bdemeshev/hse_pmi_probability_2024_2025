% arara: xelatex
\documentclass[12pt]{article}

% \usepackage{physics}

\usepackage{hyperref}
\hypersetup{
    colorlinks=true,
    linkcolor=blue,
    filecolor=magenta,      
    urlcolor=cyan,
    pdftitle={Overleaf Example},
    pdfpagemode=FullScreen,
    }

\usepackage{tikzducks}

\usepackage{tikz} % картинки в tikz
\usetikzlibrary{shapes, arrows, positioning}
\usepackage{microtype} % свешивание пунктуации

\usepackage{array} % для столбцов фиксированной ширины

\usepackage{indentfirst} % отступ в первом параграфе

\usepackage{sectsty} % для центрирования названий частей
\allsectionsfont{\centering}

\usepackage{amsmath, amsfonts, amssymb} % куча стандартных математических плюшек

\usepackage{comment}

\usepackage[top=2cm, left=1.2cm, right=1.2cm, bottom=2cm]{geometry} % размер текста на странице

\usepackage{lastpage} % чтобы узнать номер последней страницы

\usepackage{enumitem} % дополнительные плюшки для списков
%  например \begin{enumerate}[resume] позволяет продолжить нумерацию в новом списке
\usepackage{caption}

\usepackage{url} % to use \url{link to web}


\newcommand{\smallduck}{\begin{tikzpicture}[scale=0.3]
    \duck[
        cape=black,
        hat=black,
        mask=black
    ]
    \end{tikzpicture}}

\usepackage{fancyhdr} % весёлые колонтитулы
\pagestyle{fancy}
\lhead{Теория вероятностей для самураев}
\chead{}
\rhead{Контрольная: резервный день}
\lfoot{}
%\lfoot{$F(0.6) = 0.73$, $F(0.4) = 0.66$, $F(-0.25) = 0.4$, $F(1.03) = 0.85$, $F(0.05) = 0.52$ }
\cfoot{}
\rfoot{}




\renewcommand{\headrulewidth}{0.4pt}
\renewcommand{\footrulewidth}{0.4pt}

\usepackage{tcolorbox} % рамочки!

\usepackage{todonotes} % для вставки в документ заметок о том, что осталось сделать
% \todo{Здесь надо коэффициенты исправить}
% \missingfigure{Здесь будет Последний день Помпеи}
% \listoftodos - печатает все поставленные \todo'шки


% более красивые таблицы
\usepackage{booktabs}
% заповеди из докупентации:
% 1. Не используйте вертикальные линни
% 2. Не используйте двойные линии
% 3. Единицы измерения - в шапку таблицы
% 4. Не сокращайте .1 вместо 0.1
% 5. Повторяющееся значение повторяйте, а не говорите "то же"


\setcounter{MaxMatrixCols}{20}
% by crazy default pmatrix supports only 10 cols :)


\usepackage{fontspec}
\usepackage{libertine}
\usepackage{polyglossia}

\setmainlanguage{russian}
\setotherlanguages{english}

% download "Linux Libertine" fonts:
% http://www.linuxlibertine.org/index.php?id=91&L=1
% \setmainfont{Linux Libertine O} % or Helvetica, Arial, Cambria
% why do we need \newfontfamily:
% http://tex.stackexchange.com/questions/91507/
% \newfontfamily{\cyrillicfonttt}{Linux Libertine O}

\AddEnumerateCounter{\asbuk}{\russian@alph}{щ} % для списков с русскими буквами
\setlist[enumerate, 2]{label=\asbuk*),ref=\asbuk*}

%% эконометрические сокращения
\DeclareMathOperator{\Cov}{\mathbb{C}ov}
\DeclareMathOperator{\Corr}{\mathbb{C}orr}
\DeclareMathOperator{\Var}{\mathbb{V}ar}
\DeclareMathOperator{\col}{col}
\DeclareMathOperator{\row}{row}

\DeclareMathOperator{\rank}{rank}

\let\P\relax
\DeclareMathOperator{\P}{\mathbb{P}}

\let\H\relax
\DeclareMathOperator{\H}{\mathbb{H}}


\DeclareMathOperator{\E}{\mathbb{E}}
% \DeclareMathOperator{\tr}{trace}
\DeclareMathOperator{\card}{card}

\DeclareMathOperator{\Convex}{Convex}
\DeclareMathOperator{\plim}{plim}

\newcommand{\cN}{\mathcal{N}}
\newcommand{\cF}{\mathcal{F}}


\newcommand{\SST}{\text{SST}}
\newcommand{\SSR}{\text{SS}^{\text{res}}}

\newcommand{\RR}{\mathbb{R}}
\newcommand{\NN}{\mathbb{N}}
\newcommand{\hb}{\hat{\beta}}
\newcommand{\dPois}{\mathrm{Pois}}
\newcommand{\dBin}{\mathrm{Bin}}

\usepackage{mathtools}
\DeclarePairedDelimiter{\norm}{\lVert}{\rVert}
\DeclarePairedDelimiter{\abs}{\lvert}{\rvert}
\DeclarePairedDelimiter{\scalp}{\langle}{\rangle}
\DeclarePairedDelimiter{\ceil}{\lceil}{\rceil}



\begin{document}

\begin{enumerate}
\item {[10]} Величины $(X_n)$ независимы и равномерно распределены на отрезке $[1; 2]$.
\begin{enumerate}
\item {[5]} Найдите предел по вероятности
\[
\plim (X_1 - 1)(X_2 - 1)(X_3 - 1) \dots (X_n - 1) / n!
\]
% \item Сходится ли последовательность дробей в пункте (а) по распределению и в пространстве $L^2$?
\item {[5]} Найдите предел по вероятности 
\[
    \plim \frac{6n - 7}{(X_1 - 2X_2)^2 + (X_2 - 2X_3)^2 + \dots + (X_{n-1} - 2X_n)^2}.
\]
\end{enumerate}

\item {[10]} Рассмотрим две последовательности биномиально распределённых случайных величин, 
\[
X_n \sim \dBin(3n, 1/(n^2 + 5)) \quad \text{и} \quad Y_n \sim \dBin(2025, n/(2n + 30)).
\]
\begin{enumerate}
    \item {[2 + 2 + 2]} К чему сходятся по распределению последовательности $(X_n)$, $(Y_n)$ и $(X_n Y_n)$?
    \item {[2 + 2]} Если возможно, приведите пример, когда последовательность $(Y_n)$ сходится по вероятности и когда она не сходится по вероятности.
\end{enumerate}

\item {[10]} Величины $X_1$, $X_2$, $X_3$ независимы и равномерно распределены на отрезке $[1;2]$.
Найдите характеристическую функцию случайной величины $Y$,
\[
Y = \begin{cases}
    X_1, \text{ если } X_1 + X_2 > 3, \\
    0, \text{ иначе.}
\end{cases}
\]

\item {[10]} Характеристическая функция величины $X$ равна $\phi(t) = 4/(4 + \exp(-2it) - 4\exp(-it))$.
\begin{enumerate}
    \item {[6]} Какое распределение имеет величина $X$?
    \item {[4]} Найдите $\E(X)$ и $\Var(X)$.
\end{enumerate}


\item {[10]} Немного сигма-алгебр для настоящего самурая!
\begin{enumerate}
    \item {[2]} Множество всех исходов равно $\Omega = \{a, b, c, d\}$. 
    Случайная величина $Y$ определена как $Y(a) = -1$, $Y(b) = 1$, $Y(c) = 0$, $Y(d) = -1$.
    Найдите сигма-алгебру $\sigma(Y + Y^2)$.
    \item {[4]} Верно ли, что $\sigma(X + Y, X) \subseteq \sigma(X, Y)$ для произвольных величин $X$ и $Y$? Докажите или приведите контр-пример.
    \item {[4]} Верно ли, что $\sigma(X, Y) \subseteq \sigma(X + Y, X)$ для произвольных величин $X$ и $Y$? Докажите или приведите контр-пример.
\end{enumerate}

% Примечание: здесь $\sigma(R)$ — минимальная сигма-алгебра, порождённая величиной $R$, а не стандартное отклонение :)

\item {[10]} Каждый день в заезде участвую только две лошади: Юлиус и Фру-фру. 
Ставки на Фру-фру принимаются с коэффициентом $3$, то есть при победе Фру-фру ставка будет возвращена в тройном размере. 
Ставки на Юлиуса принимаются с коэффициентом $3/2$.
Безусловная вероятность победы Фру-фру равна $1/3$. 

Игрок начинает со стартовой суммой $S_0 = 100$ и каждый день ставит все свои деньги в некоторой пропорции на Фру-фру и Юлиуса. 
Игрок заметил, что условная вероятность победы Фру-фру равна $0.5$, если идёт дождь. 
Дождь идёт каждый день независимо от других с вероятностью $0.1$.

Определим долгосрочную процентную ставку $r$ условием $\plim (S_n / S_0)^{1/n} = 1 + r$, где $S_n$ — благосостояние игрока после $n$ дней.
\begin{enumerate}
    \item {[4]} Какая стратегия максимизирует $\E(S_n)$?
    \item {[4]} Какая стратегия максимизирует долгосрочную процетную ставку?
    \item {[2]} Чему равна максимально достижимая долгосрочная процентная ставка?
\end{enumerate}

\end{enumerate}
    

\end{document}

