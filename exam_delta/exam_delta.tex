% arara: xelatex
\documentclass[12pt]{article}

% \usepackage{physics}

\usepackage{hyperref}
\hypersetup{
    colorlinks=true,
    linkcolor=blue,
    filecolor=magenta,      
    urlcolor=cyan,
    pdftitle={Overleaf Example},
    pdfpagemode=FullScreen,
    }
\urlstyle{same}

\usepackage{tikzducks}

\usepackage{tikz} % картинки в tikz
\usepackage{microtype} % свешивание пунктуации

\usepackage{array} % для столбцов фиксированной ширины

\usepackage{indentfirst} % отступ в первом параграфе

\usepackage{sectsty} % для центрирования названий частей
\allsectionsfont{\centering}

\usepackage{amsmath, amsfonts, amssymb} % куча стандартных математических плюшек

\usepackage{mathtools}
\usepackage{comment}

\usepackage[top=2cm, left=1.2cm, right=1.2cm, bottom=2cm]{geometry} % размер текста на странице

\usepackage{lastpage} % чтобы узнать номер последней страницы

\usepackage{enumitem} % дополнительные плюшки для списков
%  например \begin{enumerate}[resume] позволяет продолжить нумерацию в новом списке
\usepackage{caption}

\usepackage{url} % to use \url{link to web}


\newcommand{\smallduck}{\begin{tikzpicture}[scale=0.3]
    \duck[
        cape=black,
        hat=black,
        mask=black
    ]
    \end{tikzpicture}}

\usepackage{fancyhdr} % весёлые колонтитулы
\pagestyle{fancy}
\lhead{Теория вероятностей, ПМИ}
\chead{}
\rhead{Экзамен, 2025-03-28}
\lfoot{Да пребудет с тобой Сила!}
\cfoot{}
\rfoot{}

\renewcommand{\headrulewidth}{0.4pt}
\renewcommand{\footrulewidth}{0.4pt}

\usepackage{tcolorbox} % рамочки!

\usepackage{todonotes} % для вставки в документ заметок о том, что осталось сделать
% \todo{Здесь надо коэффициенты исправить}
% \missingfigure{Здесь будет Последний день Помпеи}
% \listoftodos - печатает все поставленные \todo'шки


% более красивые таблицы
\usepackage{booktabs}
% заповеди из докупентации:
% 1. Не используйте вертикальные линни
% 2. Не используйте двойные линии
% 3. Единицы измерения - в шапку таблицы
% 4. Не сокращайте .1 вместо 0.1
% 5. Повторяющееся значение повторяйте, а не говорите "то же"


\setcounter{MaxMatrixCols}{20}
% by crazy default pmatrix supports only 10 cols :)


\usepackage{fontspec}
\usepackage{libertine}
\usepackage{polyglossia}

\setmainlanguage{russian}
\setotherlanguages{english}

% download "Linux Libertine" fonts:
% http://www.linuxlibertine.org/index.php?id=91&L=1
% \setmainfont{Linux Libertine O} % or Helvetica, Arial, Cambria
% why do we need \newfontfamily:
% http://tex.stackexchange.com/questions/91507/
% \newfontfamily{\cyrillicfonttt}{Linux Libertine O}

\AddEnumerateCounter{\asbuk}{\russian@alph}{щ} % для списков с русскими буквами
\setlist[enumerate, 2]{label=\asbuk*),ref=\asbuk*}

%% эконометрические сокращения
\DeclareMathOperator{\Cov}{\mathbb{C}ov}
\DeclareMathOperator{\Corr}{\mathbb{C}orr}
\DeclareMathOperator{\Var}{\mathbb{V}ar}
\DeclareMathOperator{\pCorr}{\mathrm{pCorr}}
\DeclareMathOperator{\col}{col}
\DeclareMathOperator{\row}{row}

\let\P\relax
\DeclareMathOperator{\P}{\mathbb{P}}

\DeclarePairedDelimiter{\abs}{\lvert}{\rvert}
\DeclarePairedDelimiter{\scalp}{\langle}{\rangle}

\let\H\relax
\DeclareMathOperator{\H}{\mathbb{H}}
\DeclareMathOperator{\plim}{plim}

\DeclareMathOperator{\E}{\mathbb{E}}
% \DeclareMathOperator{\tr}{trace}
\DeclareMathOperator{\card}{card}

\DeclareMathOperator{\Convex}{Convex}

\newcommand \cN{\mathcal{N}}
\newcommand \dN{\mathcal{N}}


\newcommand \RR{\mathbb{R}}
\newcommand \NN{\mathbb{N}}

\newcommand{\dBern}{\mathrm{Bern}}
\newcommand{\dExpo}{\mathrm{Expo}}
\newcommand{\dBin}{\mathrm{Bin}}
\newcommand{\dGamma}{\mathrm{Gamma}}
\newcommand{\dBeta}{\mathrm{Beta}}
\newcommand{\dPois}{\mathrm{Pois}}



\begin{document}



\begin{enumerate}

    \item Известно, что $\E(Y \mid X) = 2 + 3X$, $\Var(X) = 9$, $\E(X) = 6$.
    \begin{enumerate}
        \item {[2 + 3]} Найдите $\E(Y)$, $\Cov(X, Y)$.
        \item {[5]} В каких пределах могут лежать $\Var(Y \mid X)$ и $\Var(Y)$?
    \end{enumerate}

    Ответы:

    \begin{enumerate}
        \item $\E(Y) = \E(\E(Y \mid X)) = 20$, $\E(XY) = \E(\E(XY \mid X)) = 147$, $\Cov(X, Y) = 27$;
        \item $\Var(Y \mid X) \geq 0$, можно, например, считать, что $Y = 2 + 3X + R$, где $R \sim \cN(0, 1)$; $\Var(Y) \geq 81$.
    \end{enumerate}



    \item Величины $U_1$, $U_2$ распределены равномерно на отрезке $[0, 1]$ и независимы.
    Определим последовательность $X_n = n^2 \cdot I[U_1 \leq 1/(n + 2)] + U_2 \cdot n/ (n+2)$.

    \begin{enumerate}
        \item {[3]} Сходится ли $(X_n)$ почти наверное и если да, то к чему?
        \item {[2]} Сходится ли $(X_n)$ по вероятности и если да, то к чему?
        \item {[2]} Сходится ли $(X_n)$ по распределению и если да, то к чему?
        \item {[3]} Сходится ли $(X_n)$ в $L^1$ и если да, то к чему?
    \end{enumerate}

    Ответы: 

    \begin{enumerate}
        \item Да, к $U_1$, на квадрате в осях $(u_1, u_2)$ можно заметить, что сходимости нет только на множестве меры 0. 
        \item Из сходимости почти наверное следует сходимость по вероятности.
        \item Из сходимости по вероятности следует сходимость по распределению.
        \item Сходимости в $L^1$ нет, так как $\E(X_n) \geq n^2/(n+2) \to \infty$.
    \end{enumerate}


    \item 
    Рассмотрим стандартный винеровский процесс $(W_t)$.
    \begin{enumerate}
        \item {[5]} Найдите $\Cov(W_1, W_7 \mid W_3)$ и $\E(W_2^2 W_4^2)$.
        \item {[5]} При каком $\alpha$ процесс $Y_t = (3 + \alpha W_t)^2 - 10t$ будет мартингалом?
    \end{enumerate}

    \begin{enumerate}
        \item $\Cov(W_1, W_7 \mid W_3) = 0$ и $\E(W_2^2 W_4^2) = 16$;
        \item $\alpha = \pm \sqrt{10}$.
    \end{enumerate}
    
    \item Улитка стартует в точке $S_0 = 7$. 
    Каждую минуту она равновероятно смещается влево или вправо на единицу. 
    \begin{enumerate}
        \item {[3]} При какой константе $\alpha$ процесс $Y_t = \sum_{k=0}^t S_k - \alpha S_t^3$ будет мартингалом?
    \end{enumerate}
    Улитка отдыхает в точках $S_0 = 0$ и $S_0 = 20$.
    Обозначим $\tau$ момент времени, когда она впервые достигнет одной из точек отдыха, $\tau = \min\{t \mid S_t \in\{0, 20\}\}$.
    \begin{enumerate}[resume]
        \item {[4]} Слепо применяя теорему Дуба, найдите $\E(S_1 + S_2 +\dots + S_\tau)$.
        \item {[3]} Аккуратно проверьте, что теорему Дуба можно было применять. 
    \end{enumerate}

    Уточнение: без доказательства можно пользоваться тем, что $\P(S_\tau = 20) = 7/20$.

    \begin{enumerate}
        \item $\alpha = 1/3$;
        \item $\E(S_1 + S_2 +\dots + S_\tau) = 819$ или $\E(S_0 + S_1 + S_2 +\dots + S_\tau) = 826$;
    \end{enumerate}

    %\item На первом шаге величина $X$ выбирается из гамма-распределения $\dGamma(r, p/(1-p))$, где $r\in \NN$ и $p \in (0, 1)$.
    %На втором шаге величина $Y$ выбирается из пуассоновского распределения с интенсивностью $X$.
    %\begin{enumerate}
    %    \item {[2]} Найдите $\E(Y \mid X)$ и $\Var(Y \mid X)$.
    %    \item {[3]} Найдите ожидание $\E(Y)$.
    %    \item {[5]} Найдите вероятность $\P(Y = k)$.
    %\end{enumerate}

    %Подсказка: $a^{b+1} \cdot \int_0^{\infty} x^b \exp(-ax) \, dx = \Gamma(b + 1)$ и $\Gamma(b + 1) = b!$ для натуральных $b$.
    
    %\lfoot{Паниковать на экзамене строго запрещено!}
    %\lfoot{Да пребудет с тобой Сила!}

    \item Величины $X_1$, $X_2$, \dots, $X_5$ независимы и экспоненциально распределены $X_i \sim \dExpo(\lambda_i)$.
    Определим  $M = \min\{X_3, X_4, X_5\}$.
    \begin{enumerate}
        \item {[3]} Как распределена величина $M$?
        \item {[3]} Найдите вероятность $\P(X_1 < X_2)$.
        \item {[4]} Найдите функцию распределения величины $L = \ln X_1 - \ln X_2$ при $\lambda_1 = \lambda_2 = 1$.
    \end{enumerate}


    Разбалловка:

    \begin{enumerate}
        \item 0б для минимума <= t раскладывается в произведение
        
        1б не указаны пар-ры экспоненциального или указаны не верно
        \item  0б не корректная формула
        
        1б не досчитан интеграл
        
        2б ответ посчитан не правильно (арифметика)
        
        1б посчитано для $\lambda_1 = \lambda_2$
        
        \item  1б не досчитан интеграл
        
        1б ответ посчитан не правильно
        
        2б цепочка вывода правильная, но ошибка в вычислениях
    \end{enumerate}


    \item Величины $X_1$, $X_2$, \dots, $X_n$ независимы и равномерно распределены на отрезке $[0, a]$,
    рассмотрим наибольшую величину $H = \max\{X_1, \dots, X_n\}$ и наименьшую величину $L = \min\{X_1, \dots, X_n\}$.
    \begin{enumerate}
        \item {[3]} Найдите $\E(L)$ любым способом. 
    \end{enumerate}
    Определим ожидание $h(a) = \E(L \cdot H)$.
    \begin{enumerate}[resume]
        \item {[5]} Выпишите уравнение, связывающее $h(a + u)$ и $h(a)$, с точностью до $o(u)$.
        % \item {[2]} Выпишите дифференциальное уравнение, которому удовлетворяет функция $h(a)$.
        \item {[2]} Укажите начальное условие, которому удовлетворяет функция $h(a)$.
    \end{enumerate}

    Разбалловка:

    \begin{enumerate}
        \item 1б посчитано для $n=2$
        
        1б правильно посчитано распределение/плотность
        
        2б арифметическая ошибка
        
        2б в знаменателе $n$, а не $n+1$
        \item  0б формула не верная
        
        1б за уравнение с производной
        
        1б за правильную идею, но в формуле не посчитаны значения
        
        2б за правильную идею, но не правильные вычисления (вероятности или интегралы)
        \item 2б за корректный ответ
    \end{enumerate}


\end{enumerate}

\end{document}

